%\scrollmode
%\documentclass[dvips,a4paper,twoside]{article}
\documentclass[dvips,a4paper]{article}
\usepackage[dvips]{graphicx,color}
% \usepackage{times}

\usepackage{html,htmllist,makeidx,enumerate}
%\usepackage{frames}

%
% use the url.sty package, by  Donald Arseneau <asnd@triumf.ca>
% to typeset email-addresses, URLs and directory paths in LaTeX ...
%
%begin{latexonly}
 \usepackage[dvips,rightbars]{changebar}
 \usepackage{l2hman}
 \usepackage{url}
 \usepackage{longtable}
 \def\email{\begingroup \urlstyle{tt}\Url}
 \def\Email#1{\email{#1}}
% \urldef\onlinedoc\url{http://www-dsed.llnl.gov/files/programs/unix/latex2html/manual/}
 \urldef\onlinedoc\url{http://www-texdev.mpce.mq.edu.au/l2h/docs/manual/}
 \urldef\onlinedocRM\url{http://www-texdev.mpce.mq.edu.au/l2h/docs/manual/}
 \urldef\EXcolors\url{http://www-texdev.mpce.mq.edu.au/l2h/crayola/}
% \urldef\CVSrepos\url{http://cdc-server.cdc.informatik.tu-darmstadt.de/~latex2html/user/}
% \urldef\CVSsite\url{http://cdc-server.cdc.informatik.tu-darmstadt.de/~latex2html/}
% \urldef\CVSlatest\url{http://cdc-server.cdc.informatik.tu-darmstadt.de/~latex2html/l2h-latest.tar.gz}
 \urldef\CVSrepos\url{http://www.latex2html.org/user/}
% \urldef\CVSsite\url{http://www.latex2html.org/}
 \urldef\CVSsite\url{http://saftsack.fs.uni-bayreuth.de/~latex2ht/}
% \urldef\CVSlatest\url{http://saftsack.fs.uni-bayreuth.de/~latex2ht/l2h-latest.tar.gz}
 \urldef\CVSlatest\url{http://www.latex2html.org/current/}
% \urldef\patches\url{http://www-dsed.llnl.gov/files/programs/unix/latex2html/}
 \urldef\patches\url{http://www.latex2html.org/current/}
% \urldef\sourceA\url{http://www-dsed.llnl.gov/files/programs/unix/latex2html/sources/}
 \urldef\sourceA\url{http://www.latex2html.org/current/}
% \urldef\sourceB\url{ftp://ftp.mpn.com/pub/nikos/latex2html-98.1.tar.gz}
 \def\sourceB{\CTANtug\CTANA}
 \urldef\sourceC\url{http://ftp.rzg.mpg.de/pub/software/latex2html/sources/}
 \urldef\CTANtug\url{http://ctan.tug.org/ctan/}
 \urldef\CTANA\path{tex-archive/support/latex2html}
 \urldef\tugURL\url{http://www.tug.org/}
 \urldef\danteURL\url{http://www.dante.de/}
% \urldef\ListURL\url{http://www.xray.mpe.mpg.de/mailing-lists/latex2html/}
 \urldef\ListURL\url{http://www.tug.org/mailman/listinfo/latex2html/}

%
%  a little hack, needed since a footnote occurs in a figure-caption
%
\def\adjustfootnote{\setcounter{footnote}{6}\footnotetext{\url{http://csep1.phy.ornl.gov/csep.html}}}

%
% These are needed for the Glossary and Index. 
%
\newenvironment{theglossary}{\begin{list}{}{\setlength{\labelwidth}{20pt}%
 \setlength{\leftmargin}{\labelwidth}\setlength\itemindent{-\labelwidth}%
 \setlength\itemsep{0pt}\setlength\parsep{0pt}\rmfamily}}{\end{list}}
\def\dotfill{\leaders\hbox to.6em{\hss .\hss}\hskip 0pt plus  1fill}%
\def\dotfil{\leaders\hbox to.6em{\hss .\hss}\hfil}%
\def\pfill{\unskip~\dotfill\penalty500\strut\nobreak\dotfil~\ignorespaces}%

\newcommand\Glossary[2]{\glossary{#1@#2}}
\newcommand{\gsl}{\textsl}
\newcommand{\indexentry}[2]{\item #1 #2}

%\newcommand{\latextohtml}{\textup{\LaTeX 2\texttt{HTML}}}%

\DeclareRobustCommand\FoilTeX{{\normalfont{\sffamily Foil}\kern-.03em{\rmfamily\TeX}}}
%
% These macros are built-in to LaTeX2HTML:
%
\newcommand{\Xy}{\leavevmode
 \hbox{\kern-.1em X\kern-.3em\lower.4ex\hbox{Y\kern-.15em}}}
\newcommand{\AmSTeX}{\protect\AmS-\protect\TeX{}}
\newcommand{\AmS}{{\protect\AmSfont A\kern-.1667em\lower.5ex\hbox{M}\kern-.125emS}}
\gdef\AmSfont{\usefont{OMS}{cmsy}{m}{n}}

\newcommand{\sameas}[1]{\ (Same as setting: #1)}
\setcounter{footnote}{0}
%end{latexonly}

\begin{imagesonly}
 \usepackage{l2hman}
\end{imagesonly}
%
\begin{htmlonly}
 \def\makeglossary{}
 \pagecolor[named]{White}
\end{htmlonly}

\input manhtml.tex


% use %sort -f -u manual.idx > manual.index for a primitive index
%
%  NOTE:  You must use LaTeX2e in order to process this document
%       If you do not have LaTeX2e, a PostScript version
%       (manual.ps) is included with this distribution.
%
%%%%%%%%%%%%%%%%%%% No changes beyond this point %%%%%%%%%%%%%%%%%%%%%%%%%%%%%

\makeindex
\makeglossary
\sloppy
%
\setlength{\textwidth}{5.5in}
%\addtolength{\oddsidemargin}{-1in}
%\addtolength{\evensidemargin}{-1in}
\setlength{\changebarwidth}{1pt}

%
% read own internals for sections/contents before any
% from the segments.
%
%\internal[sections]{}
%\internal[contents]{}

\internal[figure]{O}
\internal[figure]{S}
\internal[figure]{M}
\internal[figure]{H}
\internal[figure]{E}%{F}

\internal[table]{O}
\internal[table]{S}
\internal[table]{M}
\internal[table]{H}
\internal[table]{E}%{F}

\internal[sections]{O}
\internal[sections]{S}
\internal[sections]{M}
\internal[sections]{H}
\internal[sections]{E}%{F}
\internal[sections]{P}
%\internal[sections]{C}

\internal[contents]{O}
\internal[contents]{S}
\internal[contents]{M}
\internal[contents]{H}
\internal[contents]{E}%{F}
\internal[contents]{P}
%\internal[contents]{C}


\internal[internals]{O}
\internal[internals]{S}
\internal[internals]{M}
\internal[internals]{H}
\internal[internals]{E}%{F}
\internal[internals]{P}
%\internal[internals]{C}

%\internal[index]{O}
%\internal[index]{S}
%\internal[index]{M}
%\internal[index]{H}
%\internal[index]{E}%{F}
%\internal[index]{P}
%%\internal[index]{C}


\begin{document}
\sloppy
%
%  TITLE-PAGE 
%
\Glossary{latex2html}{\LaTeX2HTML}{}
\title{The \LaTeX2HTML{} Translator}
\author{Nikos Drakos\\Computer Based Learning Unit\\University of Leeds.}
\date{\today}
\index{Computer~Based~Learning~Unit!University of Leeds}%
\maketitle 

\htmlrule
%
\begin{center}{
Documentation revised and updated for \textsc{v97.1} and \texttt{HTML}~3.2;\\
and further revisions for \textsc{v98.1} and later, and for \texttt{HTML}~4.0 by:}
\end{center}
\medskip
\begin{center}
%begin{latexonly}
\begin{large}
%end{latexonly}
\RossMoore\\
Mathematics Department\\
\Macquarie, Sydney.
%begin{latexonly}
\end{large}
%end{latexonly}
\end{center}
\bigskip
\htmlrule

%
%  for printed version only
%
\begin{latexonly}
\begin{center}
{\large 
This document accompanies \latextohtml{} version 99.1 (beta)%
\footnote{\bfseries This is a preliminary document for \latextohtml{} 99.1.}}.

\paragraph{}
This manual differs from earlier versions by updates to the section
\htmlref{``Installation and Further Support''}{sec:sup},
a few small changes in other sections,
and a shortening of \htmlref{``Known Problems''}{sec:prb} 
by removing references to old problems that no longer occur.
Not all of the newer features of \latextohtml{} \textsc{v99.1} are described yet. 
A fully updated manual is underway and will be released when completed.

\paragraph{History}
\NikosDrakos' original manuscript was updated for version \textsc{v96.1}\,%,
by \HerbSwan\ and converted for \LaTeXe{} by \Goossens.
Extensive revisions were made by \RossMoore\ for \textsc{v96.1} \texttt{rev-f}, 
incorporating also suggestions from \Goossens.
Another major revision was required to adequately describe the new features 
made possible with \texttt{HTML} 3.2\,,
and recent developments in image-generation and macro-handling.
This work was done by \RossMoore, 
as were most of the revisions for \textsc{v98.1}, \textsc{v98.2}
and \textsc{v99.1}.

Portability for non-Unix systems has been achieved due to work done mainly by
\Rouchal, \Wortmann, \Popineau{} and \Taupin.

Changes for the \textsc{v98.1} revision are indicated with narrow change-bars.
Wider change-bars indicate where the most recent changes and features 
added in \textsc{v99.1} are described.
\Glossary{latex2e}{\LaTeXe}%
\end{center}
\end{latexonly}

%
%  for HTML version only
%
\begin{htmlonly}

This document accompanies \latextohtml{}\Glossary{latex2html}{\LaTeX2HTML}{},
version 99.1(beta).
\Glossary{latex2e}{\LaTeXe}{}

This manual differs from earlier versions by updates to the section
\htmlref{``Installation and Further Support''}{sec:sup},
a few small changes in other sections,
and a shortening of \htmlref{``Known Problems''}{sec:prb} 
by removing references to old problems that no longer occur.
Not all of the newer features of \latextohtml{} \textsc{v99.1} are described yet. 
A fully updated manual is underway and will be released when completed.

\textbf{History: }
The manuscript was updated for version 96.1 by \HerbSwan{} 
and converted for \LaTeXe{} by \Goossens.

Updates and extensive revisions to the manuscript for version \textsc{v96.1}
\texttt{rev-f}, were made by \RossMoore,
also incorporating suggestions from \Goossens. 

\bigskip
Another major revision was required to adequately describe the new features 
made possible with \texttt{HTML} 3.2\,,
and developments in image-generation and macro-handling.
This work was done by \RossMoore, 
as were most of the revisions for \textsc{v98.1}, \textsc{v98.2} and \textsc{v99.1}.
Portability for non-Unix systems has been achieved due to work done mainly by
\Rouchal, \Wortmann, \Popineau{} and \Taupin.

Appropriate change-bar icons indicate where newly added features are described.

\htmlrule

%
% customise the URLs below, for the nearest copies of  manual.ps(.gz)
%
\index{PostScript version}
A \htmladdnormallink{PostScript version}%{manual.ps}
{http://www-texdev.mpce.mq.edu.au/l2h/docs/manual.ps.gz}
is available. %; also with \htmladdnormallink{no included fonts}%{manual.ps.gz}
%{ftp://www-texdev.mpce.mq.edu.au/l2h/docs/manual.nofonts.ps.gz},
%instead with references to PostSript fonts.

\htmlrule

Browse the following links for information concerning:

\begin{htmllist}\htmlitemmark{RedBall}
\item[Contributions from others --- \htmlref{early development}{credits}] 
\item[Contributions from others --- \htmlref{recent developments, 1996}{recent96}] 
\item[Contributions from others --- \htmlref{recent developments, 1997}{recent97}] 
\item[Contributions from others --- \htmlref{recent developments, 1998}{recent98}] 
\item[Proposals for \htmlref{future development}{future}] 
\item[\htmlref{Licensing}{licence}] 
\end{htmllist}

\htmlrule

\textbf{Warning:} The contents of this document are likely to change.\\
It is advisable not to use links to any pages other than the first page (i.e. this page).

\end{htmlonly}

\htmlrule
\latex{\newpage\vglue1pt\vfil}
%
%
%  ABSTRACT
%
\Glossary{latex}{\LaTeX}{}%
\Glossary{perl}{\textsl{Perl}}{}%
\glossary{HTML}%
\begin{abstract}%
\latextohtml{} is a conversion tool that allows documents
written in \LaTeX{}  to become part of the World-Wide Web.
In addition, it offers an easy migration path towards
authoring complex hyper-media documents using
familiar word-processing concepts, including the power of a \LaTeX-like
macro language capable of producing correctly structured \texttt{HTML} tags.

\latextohtml{} replicates the basic structure of a \LaTeX{}  document 
as a set of interconnected \texttt{HTML} files which can be explored using
automatically generated navigation panels. 
The cross-references, citations, footnotes, the table-of-contents and the lists
of figures and tables, are also translated into hypertext links. Formatting
information which has equivalent ``tags'' in \texttt{HTML} 
(lists, quotes, paragraph-breaks, type-styles, etc.) 
is also converted appropriately. 
The remaining heavily formatted items
such as mathematical equations, pictures etc. are converted to images
which are placed automatically at the correct position in the
final \texttt{HTML} document.

\latextohtml{} extends \LaTeX{}  by supporting arbitrary hypertext links 
and symbolic cross-references between evolving 
remote documents. It also allows the specification
of \emph{conditional text} and the inclusion of raw \texttt{HTML} commands.
These hyper-media extensions to \LaTeX{}  are available as 
new commands and environments from within a \LaTeX{}  document.

This document presents the main features of \latextohtml{} and
describes how to obtain and install it, and how to use it effectively.
\end{abstract}


%
%  CREDITS
%
\latex{\pagenumbering{roman}}
\clearpage\section*{Credits, 1993--1994\label{credits}}%
\index{Computer~Based~Learning~Unit!University of Leeds}\html{\\}%
Several people have contributed suggestions, ideas, solutions, support
and encouragement. Some of these are \RodWilliams, \AnaPaiva, 
\JamilSawar\ and \AndrewCole\ at the \CBLU.

\begin{htmllist}
\htmlitemmark{YellowBall}
\index{CERN!World-Wide Web Project}%
\item [\CERN]
The idea of splitting \LaTeX{}  files
into more than one component, connected via hyperlinks, 
was first implemented in \Perl{} by Toni Lantunen at CERN.
Thanks to \RobertCailliau{} of the World-Wide Web Project, also at CERN, 
for providing access to the source code and documentation 
(although no part of the original design or the actual code has been used).


\item [\RobertThau]
has contributed the new version of \fn{texexpand}. 
Also, in order to translate the ``document from hell'' (!!!) 
he has extended the translator to handle \Lc{def} commands, 
nested math-mode commands, and has fixed several bugs.

\item [Phillip Conrad and L. Peter Deutsch.]
The \fn{pstogif} \Perl{} script uses the \fn{pstoppm.ps} \PS\ program, 
originally written by Phillip Conrad (Perfect Byte, Inc.) and 
modified by L. Peter Deutsch (Aladdin Enterprises).

\item [\RodWilliams]
The idea of using existing symbolic labels to provide cross-references
between documents was first conceived during discussions with Roderick.


\item [\EricCarroll] 
who first suggested providing a command like \Lc{hyperref}\,. 

\index{accents!foreign}%

\item [\FranzVojik]
provided the basic mechanism for handling foreign accents.


\item [\ToddLittle]
The \Cs{auto\_navigation} option was based on an idea by Todd.


\item [\AxelBelinfante]
provided the \Perl{} code in the \fn{makeidx.perl} file, 
as well as numerous suggestions and bug-reports. 

\item [\VerenaUmar] (from the \CSEP) 
has been a very patient tester of some early versions of \latextohtml{}
and many of the current features are a result of her suggestions. 

\index{mailing list!Argonne National Labs}%

\item [Ian Foster and Bob Olson.]
Thanks to \IanFoster{} and \BobOlson{} at the Argonne National Labs, 
for setting up the \maillist.
\end{htmllist}


\clearpage
\section*{Later Developments, 1995--1996\label{recent96}}%
%
Since 1995 the power and usefulness of \latextohtml{} has been enhanced significantly.
The revisions later than \textsc{v95.1} have been largely due 
to the combined efforts of many people, other than the original author.
Interested users have supplied patches to fix a fault, 
or implement a feature that previously was not supported. 
Often a question or complaint to the discussion-group 
(see \hyperref{Getting Support ...}{Section~}{}{support})
has spurred someone else to provide the necessary ``patch''.%

\bigskip\noindent
Arising from this work, special credit is due to: 
\begin{htmllist}
\htmlitemmark{GreenBall}
\item [\Hennecke] for his many extensive revisions; 

\item [\Noworolski] for coordinating \textsc{v95.3};
 
\item [\Isani] for his improvement in GIF quality; 

\index{CERN!World-Wide Web Project}\index{CERN!Michel Goossens}%
\item [\Goossens]
was the driving force behind the upgrade to \LaTeXe{} compatibility,
and other features developed at CERN;

\item [Herb Swan]
for coordinating \textsc{v96.1} of \latextohtml, 
including much of the \Perl{} code 
for the new features that were introduced,
and for providing a series of bug-fix revisions 
prior to  \textsc{v96.1} \texttt{rev-f};

\item [\RossMoore]
who has revised and extended this manual, helped design and test the 
segmentation strategy, and later revisions of \textsc{v96.1}\,.
Ross organised the release of \textsc{v96.1} \texttt{rev-g} 
and provided many of the improvements
incorporated into \textsc{v96.1} \texttt{rev-h}. 

\item [\Wilck]
for the initial work on implementation of \env{frames}. 
Also Martin did most of the work implementing the extensive citation and
bibliographic features of the \env{natbib} package, written by \PatrickDaly.
He also provided the \fn{makeseg} \Perl{} script to create Makefiles
for segmented documents.

\item [\Lippmann]
for organising the releases \textsc{v96.1} \texttt{rev-h} to \textsc{v98.1}. 
Jens made significant contributions to
the internal workings of \latextohtml, 
as well as cleaning up much of its source code. 
\end{htmllist}

\htmlrule


\bigskip\noindent
Many others, too many to mention, contributed bug-reports, 
fixes and other suggestions.

\latex{\bigskip}\htmlrule

\index{URL!url package@\env{url} package}%
\noindent
Thanks also to \Arseneau{} for allowing his \fn{url.sty} 
to be distributed with this manual. 
Similarly, thanks to \JohannesBraams{} for \fn{changebar.sty}.
%
\index{change-bars}\html{\\}
Both of these are useful utilities which enhance the appearance of the printed manual. 
\begin{latexonly}
In particular, changes introduced with  \textsc{v98.1} and its revisions are denoted 
by thin change-bars, while thicker bars denote changes introduced with  \textsc{v98.2}
and later releases.%
\end{latexonly}



\clearpage
\section*{Developments: late 1996 to mid 1997\label{recent97}}%
%
During the latter part of 1996 there was much work on improving the
capabilities of \latextohtml.
Some of this was due to the \WiiiC's proposals for \HTMLiii, 
becoming a formal recommendation in November 1996,
and their subsequent acceptance in January 1997. 
Existing \LaTeX{} markup for effects such as centering, left- 
or right-justification of paragraphs,
flow of text around images, table-layout with formal captions, etc.
could now be given a safe translation into \HTMLiii, compliant with a standard
that would guarantee that browsers would be available to view such effects.  

At the same time developers were exploring ways to enhance the overall
performance of \latextohtml.
As a result the current \textsc{v97.1} release has significant improvements in
the following areas:
%
\begin{htmllist}\htmlitemmark{OrangeBall}
% 
\item[image-generation]
is much faster, requires less memory
and inline images are aligned more accurately; 
%
\item[image quality]
is greatly improved by the use of anti-aliasing effects for on-screen clarity,
in particular with mathematics, text and line-drawings; 
%
\item[memory-requirements]
are much reduced, particularly with image-generation;
%
\item[mathematics]
can now be handled using a separate parsing procedure;
images of sub-parts of expressions can be created, rather
than using a single image for the whole formula;
%
\item[macro definitions]
having a more complicated structure than previously allowed,
can now be successfully expanded;
%
\item[counters]
and numbering are no longer entirely dependent on the \texttt{.aux}
file generated by \LaTeX;
%
\item[decisions]
about which environments to include or exclude can now be made;
%
\item[HTML effects]
for which there is no direct \LaTeX{} counterpart
can be requested in a variety of new ways;
%
\item[HTML code]
produced by the translator is much neater and more easily readable,
containing more comments and fewer redundant breaks and \HTMLtag{P} tags.
%
\item[error-detection]
of simple \LaTeX{} errors, such as missing or unmatched braces, 
is now performed --- a warning message shows a line or two
of the source code where the error has apparently occurred;
%
\end{htmllist} 


\medskip\htmlrule[50\% center]
\noindent
For these developments, thanks goes especially to: 
%
\begin{htmllist}
%
\item [\Lippmann]
for creating and maintaining the CVS repository at \CVSrepos\,. 
This has made it much easier for the contributions from different developers
to be collected and maintained as a ``development version'' which
is kept up-to-date and available at all times. Together with \Rouchal\
he produced an extensive rewrite of the \fn{texexpand} utility.


\item [\RossMoore]
for extensive work on almost all aspects of the \latextohtml{} source
and documentation,
combining code for \LaTeX{}, \Perl{}, \texttt{HTML} and other utilities.
Most of the coding for the new features based on \texttt{HTML} 3.2, 
many of the new packages, faster image-generation 
and the improved support for mathematics 
and other environments, is his work.


\item [\Rouchal]
for extending the former \fn{pstogif} utility,
transforming it into \fn{pstoimg} which now allows for 
alternative image formats, such as \fn{PNG}.
Also he produced the neat \fn{configure-pstoimg} script, which eases
\latextohtml{} installation, and a rewrite of \fn{texexpand}.



\index{portability!Unix systems}\index{latex2html-NG}%
\item [\Hennecke]
who has always been there, up-to-date with developments in \texttt{HTML} and
related matters concerning Web publishing, 
and tackling the issues involved with portability 
of \latextohtml{} to Unix systems on various platforms.

Furthermore Marcus has produced \latextohtmlNG, a version of
\latextohtml{} which handles expansion of macros in a more ``\TeX-like''
fashion. This should lead to further improvements in speed and efficiency,
while allowing complicated macro definitions to work as would be expected
from their expansions under \LaTeX.
(This requires \Perl{ 5}\,, 
using some programming features not available with \Perl{ 4}\,.)%


\item [\Popineau] has produced
an adaptation for the Windows NT platform, of \latextohtml{} \textsc{v97.1}\,.

\item [\Wortmann]
showed how to configure \appl{Ghostscript} to produce
anti-aliasing effects within images.

\item [\AxelRamge]
for various suggestions and examples of enhancements,
and the code to avoid a problem with \appl{Ghostscript}.

\end{htmllist}

\medskip\vfil\htmlrule\bigskip\noindent
Thanks also to all those who have made bug-reports, supplied fixes 
or offered suggestions as to features that might allow \latextohtml{} 
to be used more efficiently in particular circumstances. 
Most of these have been incorporated into this new version \textsc{v97.1}\,,
though perhaps not in the form originally envisaged.
Such feedback has contributed enormously to helping make \latextohtml{} the
easy to use, versatile program that it has now become.

\bigskip
\begin{center}
Keep the ideas coming!
\end{center}
\bigskip
\vfil




\subsection*[center]{1st \LaTeX2HTML{} Workshop\\Darmstadt, 
15 February 1997\label{darmstadt}}
Thanks again to \Lippmann\ and members of the \LiPS\ for organising this meeting; 
also to the \FIDarmstadt\ at \Darmstadt\ for use of their facilities.

\noindent
This was an opportunity for many of the current \latextohtml{} developers to
actually meet for the first time; rather than communication by exchange
of electronic mail messages.
%
\begin{itemize}
\item
\NikosDrakos\ talked about the early development of \latextohtml, while\dots
\item \dots
\RossMoore, \Lippmann\ and \Rouchal\ described recent improvements. 
\item
\Goossens\ presented a list of difficulties encountered with earlier
versions of \latextohtml{}, and aspects requiring improvement.
Almost all of these now have been addressed in the \textsc{v97.1} release,
so far as is possible within the bounds inherent in the \HTMLiii\ standard.
\item
\KrisRose\ showed how it is possible to create \texttt{GIF89} animations 
from pictures generated by \TeX{} or \LaTeX{}, using the \XypicDK\ graphics 
package and extensions, developed by himself and \XypicAUS.
\end{itemize}

\noindent
Also present were representatives from the \DANTE\ \Praesidium\ and 
members of the \LaTeXiii\ development team.\html{\\} 
In all it was a very pleasant and constructive meeting.

\vfil

\subsection*[center]{TUG'97 --- Workshop on \LaTeX2HTML\\
University of San Francisco, 28 July 1997\label{tug97}}

\noindent
On the Sunday afternoon (2.00pm--5.00pm)
immediately prior to the TUG meeting, there will be a workshop
on \latextohtml, conducted by \RossMoore\footnote{%
Mathematics Department, Macquarie University, Sydney, Australia}.

\begin{itemize}
\item[]
Admission: \$50, includes a printed copy of the latest \latextohtml\ manual.
\end{itemize}

\bigskip

\subsection*[center]{\TeX{}Northeast TUG Conference, \TeX/\LaTeX{} Now\\
March 22--24, 1998, New York City}

\noindent
Includes a workshop/presentation by \RossMoore\footnote{%
Mathematics Department, Macquarie University, Sydney, Australia}.


\subsection*[center]{Euro-\TeX{}'98, 10th European \TeX{} Conference\\
St. Malo, France --- 29--31 March, 1998}

\noindent
Several of the \latextohtml{} developers will be present.
All European (and other) \latextohtml{} users are encouraged to attend.


\vfil


\clearpage
\section*{Developments: late 1997 to early 1998\label{recent98}}%
Much of the work contributed to \latextohtml{} during this time was
related to bug fixing and maintaining the 97.1 release, in order to
reach a more stable and reliable version which produces \fn{HTML} code
conforming to the W3C standards/drafts.
To keep in context with this view, support for \texttt{HTML 4} has been
incorporated into the translator.

\smallskip\noindent
There have been improvements to the way math code is handled, as well
as font-changing and numbering commands. These now are expected to
work much closer to the way that \LaTeX{} handles them.

\smallskip\noindent
Furthermore, missing \LaTeX{} style translations for basic \LaTeX{} 
and \AmSTeX{} document classes were added to the distribution: 
\texttt{book.perl}, \texttt{report.perl}, \texttt{article.perl},
\texttt{letter.perl}, \texttt{amsbook.perl} and \texttt{amsart.perl}.
New styles implementing \LaTeX{} packages include \texttt{seminar.perl},
\texttt{inputenc.perl} and \texttt{chemsym.perl} naming but a few.

The aim is ultimately to support all \LaTeX{}, \AmSTeX{} etc. packages in the 
standard \LaTeX{} distribution, or for which there is published documentation. 
At the time of writing this aim has not quite been reached.
To support internationalisation, \Perl{} extensions were provided for
\texttt{HTML} output conforming to ISO-Latin 1, 2, 3, 4, 5, 6,
and \Unicode{} encodings.

\smallskip\noindent
All of the above work was done by \RossMoore.
\bigskip

Additional document formats are now supported, these are Indic\TeX{},
\FoilTeX{}, and \texttt{CWEB} documents.
You may use any of these packages to translate such documents together
with \latextohtml{}, refer to the instructions in the various
\texttt{README} files.

\smallskip\noindent
Thanks go to \RossMoore{} for \IndicHTML{}, to \Veytsman{} for 
\FoilTeX{}/\texttt{HTML} and to \Lippmann{} for the \texttt{CWEB} to
\texttt{HTML} translator.

\bigskip
Numerous discussions and efforts have been undertaken to get
\latextohtml{} working independent from the underlying operating
system.
Yet all obstacles are not quite taken, but it is forseeable that we are
OS independent very soon.
This release has been reported to run on OS/2, DOS, and MacOS, besides
Unix-like operating systems.
A former version has also been ported to Amiga OS, but that results
still need to be re-integrated into the source.
Ports for Windows'95 and Windows NT exist, but are not yet
integrated with the main distribution.

\smallskip\noindent
Thanks go to \Hennecke, \AxelRamge, \Rouchal{} and \Wortmann{} for
fruitful and refreshening discussions about that \fn{Override.pm}
loading scheme (which finally made its way after enough chickens and
eggs chased one another to death \mbox{$\mathsmiley$}\,),
and to \Taupin{} for his successful efforts to get \latextohtml{}
running on DOS.

\smallskip\noindent
Thanks go also to \Popineau{} for his port to Windows NT
\footnote{\ctanURL{systems/win32/web2c/l2h-win32.tar.gz}},
and \NikosDrakos{} for a Windows 95 port based on \textsc{v96.1}h
\footnote{\url{ftp://ftp.mpn.com/pub/nikos/latex2html96.1-h-win32.tar.gz}}
(which is mentioned here at last, but not least).

\medskip\noindent
We want to take the opportunity to thank \Nelson{} and the people at
Lawrence Livermore National Laboratory who help to keep up the
\latextohtml{} main archive and the mailing list, and to \Bohnet{} at
the Max Planck Institut fuer extraterrestrische Physik, Garching for
maintaining the list's online archive.
Finally thanks and greetings to all people that contributed to this
release and have not been mentioned here...

\medskip\noindent
You all showed spirit and favour. Thank you for your efforts!

\medskip
\begin{flushright}
\dots\ and don't forget Jens and the LiPS team at Darmstadt!
\end{flushright}

\clearpage
\section*{1998 to 1999}\label{recent99}%
During this period large parts of \latextohtml{} have been overhauled and
compatibility with Perl 4 broken once and for all. The 99.2 release is the
first known to work out of the box on several UNIX systems as well as on 
Windows 95, 98, NT and OS/2. The number of supported \LaTeX\ packages is
bigger than ever.

\medskip\noindent
Thanks to \Perbandt{} for testing every second alpha/beta release of
99.2 on OS/2 and ensuring that things work ok there.


\clearpage
\subsection*{Proposals for Future Development:\label{future}}%



\subsection*{\latextohtmlNG}
Developed by \Hennecke\label{latex2htmlNG} this is a version of 
\latextohtml{} that addresses various issues,
not currently handled in the best way by version \textsc{v97.1}\,.
These include:
\begin{itemize}
\item validating the \texttt{HTML} output,
so that only correctly nested tags, and their contents,
can be produced by the translator;
%
\item more \TeX-like order of macro-expansion,
so that macros and their expansions will produce exactly
the results expected from the \TeX{} implementation of \LaTeX;
%
\item faster processing,
by streamlining some of the current \Perl{} code, and allowing
shorter strings to be handled at any given time;
%
\item customisation issues,
allowing easier portability to Unix-like environments on
other platforms.
%
\end{itemize}
Many of these features have been the inspiration for new code
written for \latextohtml~\textsc{v98.1}.

\medskip\noindent
The current version of \latextohtmlNG{} can be obtained from
the developer's
\hyperref[page]{repository}{repository, see page~}{}{cvsrepos},
in the directory
\url{http://saftsack.fs.uni-bayreuth.de/~latex2ht/ng-user}.\html{\\}
Beware that the files there are \emph{not} compatible with those of the
same name that come with the current version of \latextohtml.



\subsection*{Extended Capabilities in Web browsers}
The following areas are the subject of active development
within the Web community. 
Limited support is available within \latextohtml{} for some of these features,
using the \texttt{-html\_version 4.0} command-line switch.
\begin{description}
%
\index{style sheets!CSS}%
\index{style sheets!DSSSL}%
\item [style-sheets: ] \htmladdnormallink{proposals}%
{http://www.w3.org/pub/WWW/TR/WD-style-970324.html}
for a flexible mechanism to allow cascading (CSS) and DSSSL, 
within \htmladdnormallinkfoot{\HTMLiv}{http://www.w3.org/pub/Markup/}.
%
\index{extended markup!XML}%
\index{XML!extended markup}%
\item [XML: ] \htmladdnormallinkfoot{eXtensible Markup Language}%
{http://www.w3.org/pub/WWW/TR/WD-xml.html}.

\index{mathematics!markup, MathML}%
\index{MathML!mathematics markup}%
\item [MathML: ] \htmladdnormallinkfoot{Mathematical Markup Language}%
{http://www.w3.org/pub/WWW/TR/WD-math-970515}.

\index{chemical!markup, CML}%
\index{CML!chemical markup}%
\item [CML: ] \htmladdnormallinkfoot{Chemical Markup Language}%
{http://www.venus.co.uk/omf/cml}.

\index{fonts!non-standard encodings}%
\item [Fonts: ] further support for non-standard font encodings.

\index{icons}%
\item [Icons: ] Alternative sets of icons for navigation buttons 
and other purposes.
\end{description}
For some background on these technologies read
\Goossens' survey article ``Hyper-activity in the Web-world''
in \textsl{CERN Computer Newsletter} 
\htmladdnormallinkfoot{No. 227}{http://wwwinfo.cern.ch/cnls/227/art\_xml.html},
and browse \AxelRamge's 
\htmladdnormallinkfoot{site}{http://www.ramge.de/ax/latex2html/latex2html.html}
for ideas on how they could be used with \latextohtml.



\endinput
















\clearpage\section*{General License Agreement and Lack of Warranty\label{licence}}%
This software is distributed in the hope that it will be useful
but \emph{without any warranty}. The author(s) do not accept responsibility 
to anyone for the consequences of using it or for whether it serves 
any particular purpose or works at all. No warranty is made about 
the software or its performance. 
 
Use and copying of this software and the preparation of derivative
works based on this software are permitted, so long as the following
conditions are met:
\begin{itemize}
\item 
The copyright notice and this entire notice are included intact
and prominently carried on all copies and supporting documentation.
\item 
No fees or compensation are charged for use, copies, or
access to this software. You may charge a nominal
distribution fee for the physical act of transferring a
copy, but you may not charge for the program itself. 
\item 
If you modify this software, you must cause the modified
file(s) to carry prominent notices (a \texttt{ChangeLog})
describing the changes, who made the changes, and the date
of those changes.
\item  
Any work distributed or published that in whole or in part
contains or is a derivative of this software or any part 
thereof is subject to the terms of this agreement. The 
aggregation of another unrelated program with this software
or its derivative on a volume of storage or distribution
medium does not bring the other program under the scope
of these terms.
\end{itemize} 
This software is made available \emph{as is}, and is distributed without 
warranty of any kind, either expressed or implied.
In no event will the author(s) or their institutions be liable to you
for damages, including lost profits, lost monies, or other special,
incidental or consequential damages arising out of or in connection
with the use or inability to use (including but not limited to loss of
data or data being rendered inaccurate or losses sustained by third
parties or a failure of the program to operate as documented) the 
program, even if you have been advised of the possibility of such
damages, or for any claim by any other party, whether in an action of
contract, negligence, or other tortuous action.

\smallskip
\index{copyright!Leeds}\html{\\}%
The \latextohtml{} translator is written by Nikos Drakos, 
Computer Based Learning Unit,  University of Leeds,  Leeds,  LS2 9JT.
Copyright \copyright 1993--1997. All rights reserved.

\smallskip
\index{copyright!Macquarie}\html{\\}%
The \textsc{v97.1} and \textsc{v98.1} revisions of the \latextohtml{} 
translator and this manual were prepared by
Ross Moore, Mathematics Department,  
Macquarie University,  Sydney~2109,  Australia.
Copyright \copyright 1996--1998. All rights reserved.











%
%  CONTENTS, 
%  Lists of figures, tables
%
\clearpage
\tableofcontents
\clearpage
\listoffigures
\listoftables
%begin{latexonly}
 \adjustfootnote
%end{latexonly}
\clearpage


%
%  MAIN MANUAL
%

%begin{latexonly}
\cleardoublepage
\pagenumbering{arabic}\setcounter{page}{1}
%end{latexonly}

\relax   %% this is important, else the next segment doesn't get processed

%%% START XTRACTFAQ (END is somewhere within that segment)
\segment{overview}{section}{Overview}
\latex{\vfil\goodbreak}
%%% START XTRACTFAQ
\segment{support}{section}{Installation and Further Support}
%%% END XTRACTFAQ
\latex{\vfil\goodbreak}
\segment{features}{section}{Environments and Special Features}
\latex{\vfil\goodbreak}
\segment{hypextra}{section}{Hypertext Extensions to \LaTeX{}}
\latex{\vfil\goodbreak}
\segment{userman}{section}{Customising the Layout of HTML pages}
\latex{\vfil\goodbreak}
%%% START XTRACTFAQ
\segment{problems}{section}{Known Problems}
%%% END XTRACTFAQ

%\segment{changes}{section}{Changes from Previous Versions
% \protect\label{sec:chg}\protect\index{changes|(}}

%
%  CHANGES
%
%\begin{htmlonly}
%\relax   %% this is important, else the next section doesn't get handled correctly
%\section{Changes from Previous Versions}
%\subsection{Changes upto v96}
\begin{changebar}
\begin{htmllist}
\htmlitemmark{GreenBall}
\item[Tables and math support for HTML 3.0]
Tables can now be specified in HTML, without having to generate a
GIF image.  Code is also available for HTML 3.0 equations, but it is not
currently turned on, since no browser can handle it (yet!)
(Courtesy of Marcus Hennecke \Email{marcush@crc.ricoh.com})

\item[Font generation at screen resolutions]
The quality of equation bitmaps is greatly improved by enabling the
\$PK\_GENERATION configuration variable.  When this is done
METAFont will be invoked through \texttt{dvips} to generate fonts
more suitable to screen viewing.  Ideally, this should be done
by setting the \texttt{mode} switch to \texttt{dvips}, but unfortunately
not all version of \texttt{dvips} support this option.  For those
that don't, a \fn{.dvipsrc} file is supplied with this distribution.
Sidik Isani (\Email{isani@cfht.hawaii.edu}) provided this 
change.

\item[Improved support for active image maps]
Both server and client-side maps are supported.  Image maps can
either be inline or external.  External maps can be associated
with a thumbnail image.  A separate script \texttt{makemap} helps
resolve external URLs in image maps.  (Herb Swan \Email{dprhws@edp.Arco.com})

\item[Improved image sharing and recycling]
The older version of image recycling often caused images to
overwrite each other due to confusion in the bookkeeping.
This has been fixed.  It is also no longer necessary to keep
generating the same image which is being used repeatedly in a
document.  Furthermore, images with thumbnails can now be
recycled, as well as active image maps.  Only images of the
correct size are recycled.  (Herb Swan)

\item[Document segmentation]
Large documents can now be divided into independently processed
segments.  Additional command line switches provide intersegment
navigation information, while automatically generated Perl
files pass symbolic references and counter information
between segments.  (Herb Swan, assisted by Michel Goossens
\Email{Michel.Goossens@cern.ch})

\item[Command parsing made more like \LaTeX's]
Constructs like \verb|\hello2| are now treated as macro
\texttt{hello} followed by argument `2', rather than as
macro \texttt{hello2}.  Added \makeatletter and \makeatother
commands.  (Marcus Hennecke)

\item[Graceful termination upon interrupt]
When \latextohtml\ is interrupted by a termination signal, all
child tasks are also terminated.  This provides a more orderly
shutdown than that offered by previous versions.  (Herb Swan)

\item[Support for textual font size changes]
HTML 2.1 support is now available fo ISO 10646
\htmladdnormallink{Unicode}{ftp://dkuug.dk/i18n/ISO_10646}
character sets by specifying \fn{-html\_version 2.1}.
HTML 3.0 support is provided for textual font size changes
(like \verb|<SMALL>|), used in conjunction wich a generated
optional style sheet.  (Marcus Hennecke)

\item[More inline math generated in \texttt{HTML}]  Small inline
equations which can be typeset in \texttt{HTML} \textbf{are} typeset
in \texttt{HTML}.  This further reduces the nuber of GIF files
that need to be generated.

\item[Hypertext references can now be external]  The commands
\texttt{htmlref} and \texttt{hyperref} now accept labels defined
in an external document if the internal reference is not found.
(Herb Swan)

\item[Additional style files are now available] These were
provided by Herb Swan:
\begin{htmllist}
\htmlitemmark{YellowBall}
\item[htmllist] Defines a fancier list environment\html{, such as this}.
(It is the same as the \texttt{description} environment in the paper version.)

\item[heqn] Redefines the \texttt{equation} environment so that
equation numbers are handled in \texttt{HTML}.  This causes 
equations in this environment to be recyclable.  It also causes
equation arrays to be recyclable if their equation numbers do not
change from the previous run.
\item[floatfig] Provides support for this environment by making it
look like an ordinary figure in the electronic version.
\item[wrapfig] (Same comment as for \texttt{floatfig}).
\item[graphics] Defines elements of the standard \LaTeX 2e 
\texttt{graphics} package.
\end{htmllist}

\item[More support for \LaTeXe] Provided support for the
\texttt{ensuremath} command.  The last \emph{option} to the \texttt{babel}
package is interpreted as the \latextohtml{} language style file to load.
User-defined commands and environments can now have an optional
argument.  Stubs have been provided for \texttt{enlargethispage}
and \texttt{suppressfloats}.  \LaTeXe packages \texttt{alltt,
graphics, graphicx, color} and \texttt{epsfig} were
provided.  This manual itself was transformed into standard
\LaTeXe.  (Herb Swan and Michel Goossens)

\item[Orientation manipulation for figures] The \texttt{flip=}
option of \texttt{htmlimage} causes the program \texttt{pnmflip} to
be called prior to the generation of the GIF file.  This allows
the electronic version of the image to be oriented differently
than the paper version.  (Herb Swan)

\item[Better treatment of null images]  If for some reason an
image produced a null GIF file, then no reference is made to that
file and the program proceeds gracefully.  Furthermore, such null
images do not need to be regenerated.  This would occur
in \texttt{HTML} 3.0 images whose caption is enclosed in a \texttt{parbox},
for example.  (Marcus Hennecke)

\item[Independent control over top/bottom navigation panels]
There are now separate subroutines to control the top and bottom
navigation panels.  (Jens Krinke \Email{krinke@ips.cs.tu-bs.de}).

\item[Labels, equations and images in section headings]
All of these items are now permitted inside a section heading,
and in the \verb|\title{}| command.
However, the equations may not look very good because of the size
difference.  (Herb Swan)

\item[Other changes] The following other numerous changes are
in no particular order:

\begin{itemize}
\item Fixed the treatment of \verb|@{}| expressions in HTML 3.0 tables.
\item Made the \texttt{reuse} command line switch accept an option.
\item Made the \texttt{.tex} filename suffix optional. (This will
save a bit of typing!)
\item Added a \fn{-debug} command line switch.
\item Support unbreakable spaces (\~{}) and itallic correction.
\item Fixed a bug in \fn{pstogif} which caused \fn{ppmquant} never
to be called.
\item The \verb|\special| \index{special command}
command can now be used outside
of a figure environment.  This is useful for specifying
a PostScript prolog to \texttt{dvips}.
\item Added the configuration variable \$TEXINPUTS to point
to the path of \LaTeX\ style files that \latextohtml{} should interpret.
\item Added the configuration variable \$DVIPS\_MODE to specify:
\begin{enumerate}
\item That \texttt{dvips} understands the \texttt{-mode} switch, and
\item Which METAFont mode to use for dynamic font generation.
\item The default output file prefix.
\end{enumerate}
\item Fixed up comment removal code to remove all comments during preprocessing.
\item Make \texttt{\$LATEX2HTMLSTYLES} a path list (and support `\~{}' as 
a way of specifying the home directory).
\item Replace double dashes with a single dash.
\item Removed extraneous spaces in comma separated citations.
\item Changed encoded `\~{}' character to a real `\~{}' character at end of conversion
\item Eliminated the spurious newline that was appended to macro
expansions.
\item Improved the error reporting of \texttt{install-test}.
\item Remove whitespace from beginning of unnumbered section names.
\item Add other language support for TOC, figures, tables, and
\verb|\today|.
\item Ignore \verb+"|+ and \verb+"-+ strings in \texttt{german} documents.
\item Make \verb|\hrule| insert less vertical whitespace.
\item Added an \verb|\htmlrule| \index{htmlrule}
command to draw a horizontal rule, even within a figure caption.
\item The \texttt{texexpand} program now ignores anything after the
\begin{verbatim}
\end{document}
\end{verbatim}
command, except when it is shielded by the \texttt{verbatim}
environment.
\item The optional caption argument is now the one which goes
	to the list of figures/tables, not the required argument.
\item Added a \verb|\LaTeXe| command.
\item Fixed bug in the treatment of the \verb|\circ| math symbol.
\item A comment that appears on the first line of an included file
	is now properly removed.
\item Added a call to \texttt{replace\_user\_references}, to allow
style files to add new types of cross\-ref\-eren\-ces.
\item Removed the expansion of \verb|`\\'| to \verb|`\\ '| in verbatim.
\item Figure numbers now appear in captions enclosed in parboxes.
\item All deferred warning messages are now correctly reported.
\item The ``\texttt{scale=}'' option of \verb|\htmlimage| now works.
\end{itemize}
\end{htmllist}
\end{changebar}
 
\subsection{Changes up to v95}
\begin{htmllist}
\htmlitemmark{RedBall}
\item[\textbf{Much improved inlined equation baseline alignment!}]
(Thanks to Mark Segal \Email{segal@spud.asd.sgi.com})
Inlined equation bitmaps are now aligned correctly depending 
on whether they contain subscripts, superscripts etc. 
\item[\textbf{Support for internationalization}]
(Thanks to Martin Boyer \Email{gamin@ireq-robot.hydro.qc.ca})
A global variable \texttt{LANGUAGE\_TITLES} can now be used to change the
language in which some section titles (eg ``Table of Contents'') are
printed. It is also very easy to add support for more languages.
\item[\textbf{Compatibility with Perl 5}]
\item[\textbf{More efficient implementation}]
There has been a major overhaul of the way the source text is parsed
and analyzed in order to reduce the memory requirements of this
process.

This has been achieved by spawning off separate Unix processes to deal
with each of the \texttt{input}'ed or \texttt{include}'d files. As each
process
terminates all the space that it used is reclaimed. 
Asynchronous communication between processes takes place using 
the Unix DataBase Management system (DBM or NDBM) 
which should be present
on your system.

\textbf{To take advantage of these changes}, 
it is necessary to split the source text 
into more than one files which can be assembled using the \LaTeX
\texttt{input} or \texttt{include} commands.


\item[\textbf{``Off-line'' Image Generation}]

Two new options \texttt{-no\_images} and \texttt{ -images\_only} allow
``off-line'' image conversion. The advantage of using these options is 
that the translation can be allowed to finish even when there are
problems with image conversion. In addition it may be possible to 
fix manually any image conversion problems and then run \latextohtml{}
again just to integrate the new images without having to translate
the rest of the text. More instructions on how to do this are
included in the ``Troubleshooting'' section of the \latextohtml{}
manual.

Can now use either the \fn{pbmplus} or the \fn{netpbm} libary.
If \fn{netpbm} is used then it is no longer necessary to get and
install \fn{giftrans} in order to generate transparent inlined 
images.

Also, a new option \texttt{map=\Meta{image map URL}} in the 
command \texttt{htmlimage} can turn an included postscript image into an active
image map.

\item[\textbf{New options}] \hfill
\begin{htmllist}
\htmlitemmark{OrangeBall}
\item [-no\_images]
Do not attempt to produce any inlined images. 
The missing images can be generated "off-line" by restarting \latextohtml{}
with the option \texttt{-images\_only}.
\item [-images\_only]
Try and convert any inlined images that were left over from previous
runs of \latextohtml. 
\item [-no\_reuse]
Do \textbf{not} reuse images generated during previous translations.
(This will enable the initial interactive session during which the user is
asked whether to reuse the old directory, delete its contents or quit)
\item [-no\_subdir]
Place the generated HTML files  in the 
current directory. The default behaviour is to create (or reuse)
another file directory.
\item [-ps\_images]
Use links to external postscript images rather than inlined GIF images.
\end{htmllist}
\item[\textbf{Several small changes and bug fixes}] \hfill
\begin{itemize}
\item It is no longer necessary to get \fn{Giftrans} if \fn{NETPBM}
is available
\item Fixed problems with support for \fn{german.sty}
\item Fixed problems with multiple bibliographies. Each bibliography
is now treated as a separate section.
\item Added support for ``dotless i's and j's''.
\item  Fixed to resolve figure and table numbers when captions contain
accented characters.
\item Added support for hierarchical indices with duplicate index keys
\item Fixed problem with HTML encodings of ISO-LATIN1 characters   
creeping into converted images of figures and tables.
\item  Added new global variable \texttt{PAPERSIZE}
to make it easier to change the default behavior when converting large images
\item A new configuration variable (\texttt{TRANSPARENT\_IMAGES}) can be 
used to stop any inlined images generated from "figure" environments
from being transparent.
\item Fixed problem which occurs when \fn{getcwd.pl} is not part of the
Perl library.
\item The option \texttt{-address ""} is now valid.
\item Fixed a problem with tables of contents.
\item Fixed problem with the use of the \texttt{finger} command when
trying to find out the name of the user.
\item Added some support for \texttt{tabbing} environments.
\item HTML heading elements no longer contain other markup.
\item Fixed problem with optional argument in citation commands
\item Added some more support for LaTeX2e.
\item Fixed problem with displayed equations forcing all the remaining
equations to be displayed.
\item Fixed bug with converting postscript images containing more than
256 colors.
\end{itemize}
\end{htmllist}

\subsection{Changes upto v0.6.2}

\begin{htmllist}
\htmlitemmark{RedBall}
\item[\textbf{Image Conversion}] \hfill
\begin{itemize}
\item The \fn{pstogif} script has been rewritten in Perl and it
now accepts more options for specifying color depth, scale factors
and pixel density.
\item LaTeX \texttt{figure} and other environments are now processed
in 24-bit color.
\item Figure and table captions are now converted into HTML
rather than becoming part of the inlined image. This means that 
any cross-references in the captions will become active.
\item It is now possible to control for each generated image:
\begin{itemize}
\item its size 
\item whether it should be inlined or left as an external image
\item if left as an external image whether it should be accessible
via a \htmlref{\textbf{generated thumbnail sketch}}{fig:example}
or a textual hypertext link
\end{itemize}
These options are available from a new LaTeX command defined in {\fn
html.sty} called \fn{htmlimage}. 

\item Equations and other inlined images are converted into
transparent GIFs rather than XBMs. \textbf{This makes it possible 
to reduced their storage requirements to about 1/6th of what was
previously required!}
\item The default size of equations and other small inlined 
images can be changed
by setting the variable \texttt{MATH\_SCALE\_FACTOR} in the
configuration files.
\item The default size of figures, tables, and other large
inlined images can be changed by setting the variable 
\texttt{FIGURE\_SCALE\_FACTOR} in the
configuration files.

\item Dvips is now called with the \texttt{-M} option which stops it
from invoking \texttt{METAFONT}.
\end{itemize}

\item[\textbf{Optimization}] \hfill
\begin{itemize}
\item Some effort has gone into reducing the amount of memory required
during conversion. \textbf{The impact of these changes can be very
significant}.
In particular, files
containing large numbers of LaTeX commands (eg \fn{.aux} or \texttt{.bib}) 
files are handled much better.
\end{itemize}

\item[\textbf{Backwards Incompatible Changes}] \hfill
\begin{itemize}
\item The option \texttt{-allbitmaps} has been removed. This is no
longer necessary as \latextohtml{} can now generate transparent
GIF images.
\item The definition of the command \texttt{htmladdnormallink} in the 
file \fn{html.sty} has changed and the URL will no longer appear
as a footnote in the paper (DVI) version. The translation of
this command into HTML is \textbf{not} affected. 

The previous functionality can be obtained with a new command
\texttt{htmladdnormallinkfoot} which will add the URL as a footnote
in the paper version.
\item The script \fn{pstoxbm} is no longer distributed as it is not
used.

\end{itemize}

\item[\textbf{Bug Fixes}] \hfill
\begin{itemize}
\item The \fn{install-test} script now recognizes version
numbers correctly and gives better warnings. It also makes
executable any scripts that need to be so. 
\item Stopped using the \texttt{nslookup} program to try and guess
e-mail addresses (they were used in signing each generated page). 
This has been replaced with a simple call 
to \texttt{finger}.
\item Fixed problem which could cause ``Parameter overflow'' in 
TeX during image conversion.
\item The \fn{texexpand} script has been changed extensively.
Some nasty looping problems are now avoided, and the tracking
down of included files has been improved.
\item Some reported incompatibilities with some Unix shells
(bash,OSF) have been fixed.
\item Displayed equations now appear correctly on separate
lines and are right justified.
\item Fixed bug with nested environments.
\item Fixed problem which caused the wrong numbers to be assigned
to sections with the same titles when the \texttt{-show\_section\_titles}
is used (thanks to Brian  Toonen \Email{toonen@mcs.anl.gov}).
\item Fixed problem with the \texttt{hyperref} command which
caused the wrong ``hyperized'' text to appear in the final document.
\item \texttt{item} optional arguments can now contain one level of
nesting eg \texttt{item[[First Choice]]}.
\item Fixed problems with the ``image caching and reuse'' mechanism 
which avoids converting images unnecessarily..
\end{itemize}

\item[\textbf{Other Changes}] \hfill
\begin{itemize}
\item LaTeX2HTML now generates \HTML{meta} HTML tags which can be used
by \htmladdnormallinkfoot{indexing scripts}
{http://www.ai.mit.edu/tools/site-index.html}
 which generate information for the \htmladdnormallinkfoot{ALIWEB}
{http://web.nexor.co.uk/aliweb/bin/aliwebsimple.pl} 
search and retrieval tool. The information in the \HTML{meta} tags
contains
the title of each separate HTML file. 
\item A new configuration variable \texttt{DEBUG} can be used to preserve
intermediate files for debugging. 
\item Some problems with respect to compatibility with the HTML2.0 
standard have been addressed. No more ``unquoted attribute value literals''.
\item All the navigation icons are now ``lynx friendly''; their
\texttt{ALT} attribute now has a meaningful value.
\item A new LaTeX command \fn{htmlref} makes it a lot simpler
to create hypertext links intended only for the HTML version of
the document.
\item \fn{DVIPSK} is now recognized by the installation script
\item Added support for the \fn{changebar.sty} file by 
David B. Johnson (dbj@titan.rice.edu).
\item It is no longer necessary to add style file names into the 
\texttt{DONT\_INCLUDE} variable as \latextohtml{} now does not
attempt to translate file included files ending in \texttt{.sty}.
\item The small invisible bitmaps that used to mark anchors have been
replaced with the invisible character \verb|&#160;|.
\item It is no longer necessary to use full path names when
including external postscript files.
\item If a file \fn{.latex2html-init} is found in the ``current
directory'' then this will be loaded automatically after loading the
other default configuration files.
\item Changed the naming convention of generated HTML files. The
``top'' document is \fn{<FILE>.html} as before but all other ``nodes''
are named \fn{nodeN.html} where \fn{N} is an integer. Also, the file
containing
the footnotes is now called \fn{footnode.html}.
\item A special extension to \latextohtml{} for those who use the 
Harvard style for references can be obtained from
\htmladdnormallink{http://www.arch.su.edu.au/\~{}peterw/latex/harvard/}{http://www.arch.su.edu.au/\~{}peterw/latex/harvard/}.
\end{itemize}
\end{htmllist}


\subsection{Changes upto v0.5.3}
\begin{itemize}
\item The files \fn{LATEX2HTMLDIR/styles/german.perl} and 
\fn{LATEX2HTMLDIR/styles/makeidx.perl} have been fixed
so that they are consistent with changes in the main script.
\item A problem with disappearing spaces after some equations
and other environments has been fixed.
\end{itemize}
\subsection{Changes upto v0.5.1}
\begin{htmllist}
\htmlitemmark{YellowBall}
\item[\textbf{Navigation Panel}]
The \hyperref{navigation panel}{navigation panel (see section
}{)}{sec:navpanel}
is now fully configurable and much better looking.
\item[\textbf{Automatic Signatures}]
The signatures at the bottom of each page are now constructed
using \fn{nslookup} (Thanks to Alberto Accomazzi
\Email{alberto@cfa.harvard.edu})
\item[\textbf{Numerous fixes including...}] \hfill
\begin{itemize}
\item ``dead'' \fn{next\_page} buttons when the \fn{-split} option
was used, and
\item internal \latextohtml{} markers appearing in section titles.
\end{itemize}
\end{htmllist}

\subsection{Changes upto v0.4}
\begin{htmllist}
\htmlitemmark{RedBall}
\item[\textbf{Accents and Special Characters}]
LaTeX accent commands, special characters and accent commands defined
in \fn{german.sty} are translated to equivalent ISO-LATIN-1
characters when that is possible (partly thanks to code supplied
by Franz Vojik \Email{vojik@de.tu-muenchen.informatik}).
\item[\textbf{Auto-loading of Style-Specific Code}]
The translator now supports a 
\hyperref{mechanism for including Perl code extensions}{mechanism for
including Perl code extensions (see Section }{)}{sec:sty}
which are specific to particular style files. 


This mechanism will help to keep the core script smaller as well as make
it easier for others to contribute and share solutions on  
how to translate specific style files. The current distribution includes the files
\fn{german.perl}, \fn{french.perl}, \fn{html.perl} and \fn{makeidx.perl}.
\item[\textbf{Installation and Testing}]
To make it easier to install the translator for more than one user
there is now a configuration file ({\fn
LATEX2HTMLDIR/latex2html.config}) which contains installation specific 
information and default values for the various options. 
This makes it unnecessary to have a \fn{HOME/.latex2html-init} for
each user.
If a user does have a  \fn{HOME/.latex2html-init} then this will be
loaded after {\fn
LATEX2HTMLDIR/latex2html.config}.

A new Perl script (\fn{install-test}) is now available with the distribution
which will install the translator, and then perform some tests on the
availability
of some external
programs giving, appropriate warnings.
\item[\textbf{Navigation Panel Extensions}]
The following new links have been added to the navigation panel which
appears in each page
(where this is appropriate):
\begin{itemize}
\item a link to the next ``logical'' page (this allows reading each
page in the same order as with a paper-based version - as opposed to
structure navigation)
\item a link to the previous ``logical'' page 
\item a link to the table of contents
\item a link to the index 
\end{itemize}
\item[\textbf{HTML Style File Extensions}]
The HTML style file \fn{html.sty} now includes definitions of new 
environments for:
\begin{htmllist}
\htmlitemmark{OrangeBall}
\item[Inclusion of Raw HTML]
A new LaTeX environment \texttt{rawhtml} allows
arbitrary
HTML tags to be included in a LaTeX document. This is useful for
taking
advantage of HTML+ features as they become available (e.g. interactive
forms).
The HTML commands are ignored when producing the DVI version of the
document.
\label{sec:cond}
\item[Conditional Text]
The new environments \texttt{latexonly} and \texttt{htmlonly} allow their
contents to appear only in final DVI or in the HTML version of a
document respectively. 

A new command \texttt{hyperref} can be used to specify the way in
which cross-references should be shown in the DVI version and the
HTML version of the document.
\end{htmllist}
\item[\textbf{Right Justification of Equations}]
The translator
adds enough whitespace at the beginning of each equation bitmap
to push it to the right-hand margin. The width of each line can be
set in the configuration file using the variable \texttt{LINE\_WIDTH}.
\item[\textbf{Inlined Images}]
All inlined images are now generated by calling \texttt{latex} and {\fn
dvips} only once
at the end of the conversion process (but each generated image still
has to be filtered individually). Also, a warning is given if an 
image cannot be converted.
\item[\textbf{Numeric Labels}]
The original (LaTeX generated) numeric labels are used instead of
the arrow navigation icon for cross-references where possible.
To do this the translator uses the information in the 
\fn{aux} file which is generated by \LaTeX. If the \fn{aux} file
is out of date a warning is given.
\item[\textbf{New Options}] \hfill
\begin{htmllist}
\htmlitemmark{RedBall}
\item[\texttt{-auto\_navigation}]
This puts a navigation panel 
at the top of each page as usual. But if the page exceeds a
user-settable  
number of words (the default is 450 words) 
then a navigation panel is also placed at the end of
the page. This option is active by default.
\item[\texttt{-index\_in\_navigation}]
Adds a link to the index in the navigation panel.
\item[\texttt{-contents\_in\_navigation}]
Adds a link to the table of contents in the navigation panel.
\item[\texttt{-next\_page\_in\_navigation}]
Adds a link to the next ``logical'' page in the navigation panel.
\item[\texttt{-previous\_page\_in\_navigation}]
Adds a link to the previous ``logical'' page in the navigation panel.
\item[\texttt{-bottom\_navigation}]
This puts a navigation panel at the bottom of each page.
\item[\texttt{-top\_navigation}]
This puts a navigation panel at the top of each page (the default).
\item[\texttt{-show\_section\_numbers}]
This allows each section (page) title to be numbered in the same 
way that it would be numbered by \LaTeX. This requires 
an up to date \fn{aux} file (generated by running \LaTeX) and that 
each title is unique. If the \fn{aux} file is not up to date then a 
warning is given.
Unfortunately if 
the title contains inlined images the numbering for that title will
be lost. 
\item[\texttt{-reuse}]
This allows images generated during previous invocations of
the translator to be ``reused'' without going through the initial 
interactive session. The same behavior is obtained by setting
the variable \texttt{REUSE} to 1 (the default) in the configuration file.
Note that images which may depend on contextual information (e.g. numerical
labels) cannot be reused and are always re-generated. 
\end{htmllist}
\item[\textbf{Bug Fixes and Minor Changes}] \hfill
\begin{itemize}
\item The links from the table of contents in single page documents (created using
the option \texttt{-split 0}) is now working as expected.
\item Newlines are translated to the \HTML{BR} tag.
\item Fixed some problems in dealing with new command macros.
\item \fn{texexpand} now respects commented \texttt{input} and 
\texttt{include} commands. Also (thanks to Franz Vojik
\Email{vojik@de.tu-muenchen.informatik}), it 
now looks in subdirectories when expanding
files.
\item The ALIGN attribute of inlined images is now BOTTOM instead of
TOP.
\item Fixed problem with the environment variable \texttt{TEXINPUTS}.
\item Added some support for optional user-defined labels (bullets) in
list environments.
\end{itemize}
\end{htmllist}

More details on all the changes are available in the file 
\fn{LATEX2HTMLDIR/Changes}.

\subsection{Changes upto v0.3.1}
These changes are mostly due to patches contributed by Robert S. Thau
\Email{rst@ai.mit.edu}:
\begin{itemize}
\item Nested environments with the same name are now dealt with
properly.
\item Commands that are passed to LaTeX for processing which have 
environments in their arguments (e.g. a \texttt{parbox} command which 
an \texttt{itemize} environment as an argument) are now
processed correctly. A general mechanism for users to 
specify the syntax of commands that should be passed to LaTeX 
is described in Page \pageref{pass}.
\item Fixed a problem with recognizing the \texttt{special} command.
\item Fixed bug in the generation of the index.
\end{itemize}

\subsection{Changes upto v0.3}
\begin{htmllist}
\htmlitemmark{YellowBall}
\item [\textbf{Image Recycling}]
Images for equations, tables, figures, special characters etc. generated by
the translator are recognized  during subsequent runs.
The user is then asked whether old images should be reused
(or whether the old images should be deleted and regenerated).
This offers tremendous improvements in speed after an initial
successful translation.

\item [\textbf{Cross-References Between (Local or Remote) Documents}]
Cross-references between 
two or more documents (possibly on remote locations) can be
established via symbolic labels
which are independent of the physical realization of these documents. 

Such cross-references will be maintained with a simple
re-translation\footnote{This is true for documents under the same
server but for remote documents a little more is required (see 
the \hyperref{example}{example in Section}{}{crossrefs})}
even after one or more of the documents have been broken into
different physical parts or moved.

The mechanism is based on the 
\hyperref{new commands}{new commands (see Section }{ )}{external}
\texttt{externallabels} and \texttt{externalref} which are an extension of the simple 
\texttt{label-ref} pairs. 

\item [\textbf{New Options}] \hfill
\begin{htmllist}
\htmlitemmark{GreenBall}
\item [\texttt{external\_images}] This provides hypertext links to where
generated images (for equations, tables, figures etc) are stored
externally. (The default is to ``inline'' generated images in the main body 
of the text.)
\item [\texttt{ascii\_mode}] This switches all the navigation icons to their
ascii equivalents. Also generated images are stored externally as with the 
\texttt{external\_images} option above. The \texttt{ascii\_mode} option
makes documents more portable as it allows them to be
viewed on browsers that do not support inlined images.
\end{htmllist}
\item [\textbf{Special Command Style File}] A style file \htmladdnormallink{\fn{html.sty}} 
{http://cbl.leeds.ac.uk/nikos/tex2html/doc/html.sty.txt} is now included in the
distribution. This contains the definitions (syntax) of some 
special LaTeX commands mainly for providing external hypertext
links.
This should be included in LaTeX files that use the any 
\hyperref{hypermedia extensions}{hypermedia extensions (see Section}{ )}{special}.

\item [\textbf{Minor Changes and Bug Fixes}] \hfill
\begin{itemize}
\item Equations, equation arrays and theorems are now numbered
correctly even when they are individually passed to LaTeX for
processing.
\item Fixed problem with \texttt{label} commands appearing in section
headings.
\item Fixed problems with verbatim environments, bibliography items,
  generated file names, the options \texttt{nolatex} and
  \texttt{no\_navigation}, the \texttt{thanks command}, the name of an
  HTML tag (\HTML{HEAD}),
\item The appearance of footnotes has been improved.
\item Some inconsistencies in the \fn{pstoxbm} and \fn{pstogif}
 scripts have been fixed.  
\item Parts of the documentation have been rewritten, restructured and
some new sections have been added. 
\end{itemize}
\end{htmllist}

\textbf{The following were contributed by Robert S. Thau
\Email{rst@ai.mit.edu}:}

\begin{htmllist}
\htmlitemmark{WhiteBall}
\item [\textbf{New Texexpand}]
This fixes problems with the standard version and 
handles the inclusion of style files which need to processed by the 
translator. Appropriate modifications to the main script were also 
made to work with new version of \fn{texexpand}. 

\item [\textbf{Handling of Raw \TeX}]
Added support for simple raw TeX commands such as  
\texttt{special} and simple instances of \texttt{def}\footnote{Where
``simple'' is defined roughly 
as ``they could have used \texttt{newcommand}, but didn't''.}. 
For the messier cases, 
the definition is scooped up and moved to the preamble.  This allows 
the translator to handle the simple, but nonstandard, postscript figure inclusion
macros as well as an awful lot of other stuff done with gratuitous 
\texttt{defs}.
\item [\textbf{More Options}] \hfill
\begin{htmllist}
\htmlitemmark{PinkBall}
\item [\texttt{dont\_include}]
This can be used  to specify style files that
should not be included in the translation.
\end{htmllist}
\item [\textbf{Other Changes}] \hfill
\begin{itemize} 
\item Added support for nested math mode expressions and general
list environments in order to handle \emph{the document from Hell!!!}.
\item Fixed problems in the translation of 
bibitems, the substitution of macro definitions, and the processing of
unrecognized commands in the preamble.
\end{itemize}
\end{htmllist}
\subsection{Changes upto v0.2}
\begin{itemize}
\item Added a command line option
to switch off the navigation links
at the top of each page.
\item The navigation icons are now part of the distribution.
\item Added a customizable separator between the main body of the text
in a page and the child links from that page.
\item The order of the navigation keywords at the top of each page is
the same as that of the navigation icons.
\item The arguments of \texttt{verbatim} environments are now translated
into fixed width fonts.
\item A warning is given if a \texttt{bbl} (bibliography) file is 
needed but not found.
\item The installation is now (mostly) done by setting variables in 
just one file.
\item The \fn{pstoxbm} script now uses environment variables
set in the initialization file.
\item Fixed bug in translating sequences of special HTML characters
(e.g. \&, $<$, etc.)
\item Fixed bug in the handling of the \verb|$$|-form of the dislay
math environment.
\item Fixed bug in the handling of the *-forms of environments.
\item Added sections on how to embed hyperlinks in a LaTeX document 
(see Page \pageref{sec:hyper}) and on how to extend the translator
(see Page \pageref{sec:extend}) in the documentation.
\end{itemize}

\subsection{Changes upto v0.1.1}
\begin{itemize}
\item Fixed bug about empty lines being inserted in environments that
cannot tolerate them (e.g. \texttt{math}).
\item Changed the format of inlined images coming back from LaTeX
from GIF to XBM. This looks better on grayscale and color monitors.
\item Fixed problem with commands being passed on to LaTeX  after
their
arguments had been translated (this affects the commands 
\texttt{psfig}, \texttt{fbox}, \texttt{framebox}, and \texttt{parbox}).
\end{itemize}

\section{Known Problems}
\index{problems} \index{bugs}
Here are some of the problems of the current version:
\begin{htmllist}
\htmlitemmark{PurpleBall}
\item [Correctness and Efficiency\index{efficiency}]
The translator cannot be guaranteed to perform as expected.
Several aspects of the implementation need
optimization and improvement. Apart from possible bugs the translator 
may place heavy demands on your resources.

\item [Unrecognized Commands and Environments \index{unrecognized commands}]
Unrecognized commands are ignored and any arguments are left in the
text. Unrecognized environments are passed to LaTeX  and the result is
included in the document as one or more inlined images.

\item [Cross-references\index{cross-references}]
References in environments that are passed to LaTeX  for processing
(e.g. a \texttt{cite}, or a \texttt{ref} command), are not processed
correctly.
\texttt{label} commands are handled correctly.

\item[Order-Sensitive Commands]
Commands which affect global parameters during the translation,
and are sensitive to the order in which they are processed may
not be handled correctly. In particular, counter manipulation
(e.g. \texttt{newcounter, setcounter, stepcounter}, etc) 
commands may cause problems.

\item [Index\index{index}]
The translator generates its own index by saving the arguments  of 
the \texttt{index} command. The contents of the \texttt{theindex}
environment are ignored.

\item[New Definitions\index{new definitions}]
New definitions (\texttt{newcommand}, \texttt{newenvironment}, 
\texttt{newtheorem} and \texttt{def}),
will not work as expected if they are defined more than once.
Only the last definition will be used throughout the document.

\item [Scope of declarations and environments]
If the scope of a declaration or environment crosses section
boundaries, then the output may not be as expected, because each
section is processed independently.

\item [Math mode font size changes]  Math mode font changes
made outside the math mode are not honored.  Thus the two equations
in
\begin{verbatim}
$a_b$ and {\LARGE $a_b$}
\end{verbatim}
would come out looking the same.  The trick is to write
\begin{verbatim}
$a_b and $\mbox{\LARGE $a_b$}$.
\end{verbatim}

\end{htmllist}

\section{Troubleshooting}
\index{debugging} \index{problems} \index{fixes}
Here are some curable symptoms:

\begin{htmllist}
\htmlitemmark{BlueBall}
\item [Cannot run any of the Perl programs]
If your Perl installation is such that Perl programs are not allowed 
to run as shell scripts you may be unable to run  \fn{latex2html}, \fn{texexpand} \fn{pstogif}
and  \fn{install-test}. In this case change the first line in each of these
programs from
\begin{verbatim}
#!/usr/local/bin/perl
\end{verbatim}

\emph{to}

\begin{verbatim}
: # *-*-perl-*-*
    eval 'exec perl -S  $0 "$@"'
    if $running_under_some_shell; 
\end{verbatim}

\item [The \fn{install-test script gives uninformative error messages}]
If for any reason you have trouble running \fn{install-test}
do not despair. Most of what it does is to do with checking
your installation rather than actually installing anything.
To do a \textbf{manual installation} just change the variable
\texttt{LATEX2HTMLDIR} in the beginning of the file \fn{latex2html}
to point to the directory where the \latextohtml{} files can be found.

Also, make sure that the files
\fn{pstogif}, \fn{texexpand} and \fn{latex2html} are executable,
and if necessary use the Unix \fn{chmod} command to make them 
executable.

\item [It just stops] Check the style
files that you are using. It is likely that you are using
a style file which contains raw TeX commands. In such a case
start \latextohtml{} with the option \texttt{-dont\_include \Meta{style file
name}}. Alternatively, add the name of the style to the variable 
\texttt{DONT\_INCLUDE} in your
\fn{HOME/.latex2html-init} file. If you don't have such a file then
create one and add the lines:
\begin{verbatim}
$DONT_INCLUDE = "$DONT_INCLUDE" . ":<style file name>";
1; 	# This must be the last line
\end{verbatim}

Another reason why \latextohtml{} might stop is that the LaTeX source
file itself contains raw TeX commands. In this case you may 
put such commands inside a 
\hyperref{\texttt{latexonly}}{\texttt{latexonly (see Section }}{)}{sec:latexonly}
environment.

\item [Perl cannot parse the \fn{latex2html} script]
Update your Perl to patch level 36. You can check which version of
Perl you are using by invoking Perl with the \texttt{-v} option.
Earlier versions of Perl than that shown above
have caused problems due
to tighter control over syntax.

\item [It crashes (dumps core) as soon as it starts \label{perl}]
Update your Perl 4 to patch level 36 or later (Perl 5).

You can check which version of
Perl you are using by invoking Perl with the \texttt{-v} option.


While you wait for your technical support people to upgrade Perl
you could try invoking Perl from within \latextohtml{} with 
the \texttt{-d} (debug) option. Then, when \latextohtml{} starts, it will
immediately fall into the Perl debugger. To continue just press 
\texttt{c <CR>}.

\item [\fn{dvips} complains about incorrect arguments \label{dvips}]
Please use a version which supports the command line options \texttt{-M -S,
-o and -i}. ``Recent'' versions at least after 5.516 do
support them.

\item [It gives an \texttt{Out of memory} message and dies] 
If you are using version \latextohtml{} 0.7 or later try splitting your 
source file into more than one files using the \LaTeX\ 
commands \texttt{input} or \texttt{include}. 
Also, use the \texttt{-no\_images} option.

As a last resort you may consider increasing the virtual memory
(swap space) of your machine. As an indication 
of what you might be able to do on your machine,
a very long book (about 1000 printed pages) required about 
24MB of RAM and over 150MB of swap space to convert on a local Sun Sparc ELC
running SunOS 4.1.3.

\item [It gives ``dbm'' related error messages]
\latextohtml{} 0.7 and later requires the
Unix DataBase Management system (DBM or NDBM) in order to run.
This is usually part of each Unix operating system but if you 
don't have it then you may need to get it. \htmladdnormallinkfoot{Use Archie}
{http://www.pvv.unit.no/archie/} to find one.

\item [The \texttt{verb"ABC"} command doesn't work]
This is a nasty bug. Please use any characters other than quotes eg
\texttt{verb+ABC+}

\item [Cannot get the ``tilde'' (\~{}) to show]
The trick here is to use the command \verb|\~{}|. 

Alternatively it is possible to use something like \\
\begin{verbatim}
\htmladdnormallink{mylink}%
  {\begin{rawhtml}http://host/~me/path/file.html\end{rawhtml}}
\end{verbatim}

or

\verb|\htmladdnormallink{mylink}{http://host/\%7Eme/path/file.html}| 

\textbf{Warning:} Some browsers may not be able to interpret the \verb|%7E|
as a ``tilde'' character.

\item [Macro definitions don't work correctly]
As mentioned in other places plain TeX definitions cannot be
converted.
But you may also have problems even when using LaTeX definitions
(with \texttt{newcommand} and \texttt{newenvironment}) if such definitions
make use of {\it sectioning or verbatim} commands. These are 
handled in a special way by \latextohtml{} and cannot be used in
macro definitions. 

In general the macro handling mechanism is inefficient and very
fragile. Avoid using macros if possible.

\item [\texttt{input} commands]
There is a bug in the expansion of \texttt{input} commands which causes a problem
when more than one \texttt{input} command appear on the same line.
There is no quick fix other than suggesting that you 
insert a newline after \texttt{input} commands in the source .tex files.

\item [\texttt{input} commands in verbatim environments]
These cause problems. There is no fix yet.

\item [Optional arguments in description environments]
If you have optional arguments for the \texttt{item} command in 
a description environment containing nested ``]'' characters then 
these may not show up correctly. To avoid the problem enclose them
in \{\}'s eg \verb+\item[{[nested [angle [brackets] are ok]]}]+

\item [LaTeX2HTML behaves differently even when you run it on the
same file]

If you notice any strange side-effects from previous runs of
\latextohtml{} try using the option \texttt{-no\_reuse} and choose 
\texttt{(d) } when prompted. This will 
clear any intermediate files generated during previous runs.
Note that this option will disable to image reuse mechanism.

\item [Cannot convert postscript images which are included
in the LaTeX file] \hfill \\
It is likely that the macros you are using for including postscript
files (e.g. \texttt{epsffile}) are not understood by \latextohtml.
To avoid this problem enclose them in an environment which will
be passed to LaTeX anyway e.g.
\begin{verbatim}
\begin{figure}
\epsffile{<postscript file name>}
\end{figure}
\end{verbatim}

Another reason why this might happen is that your shell 
environment variable 
\texttt{TEXINPUTS} is undefined. This is not always 
fatal but if you have problems you can use full
pathnames for included postscript files (even when the postscript
files are in the same directory as the LaTeX source file).
Alternatively try setting TEXINPUTS to ".::". 
With some TeX and LaTeX installations setting TEXINPUTS to 
".::" may cause problems in the normal operation of LaTeX.
If you get errors such as LaTeX complaining that it can no longer find
any style files then you must set TEXINPUTS to 
\verb|"<path to your LaTeX installation>:."|
if you want to use both LaTeX and LaTeX2HTML.

\item [Some of the inlined images are in the wrong places]
This happens when any one of the inlined images is more than a page
(paper page) long. This is sometimes the case with very large tables
or large postscript images. In this case you can try specifying 
a larger paper size (eg ``a3'', ``a2'' or even ``a0'') instead of
the default (``a4'') using the LaTeX2HTML variable \fn{PAPERSIZE} 
in the file \fn{latex2html.config}.

Another reason why this may happen is that by default the \fn{dvips} program
reverses the postscript pages it generates. If your \fn{dvips
program}
behaves in this way try changing the line
\verb|$DVIPS = "dvips";| 

to

\verb|$DVIPS = "dvips -r0";|

in the file \fn{latex2html.config}.

\item [\textbf{Unacceptable quality of converted images}]
Try changing the size of the image 
(\hyperref{See image conversion}{See Section }{}{imgcon}).

\item [The bibliographic references are missing]
Run \texttt{latex} and then \texttt{bibtex} on the original source file in
order to generate a \texttt{bbl} file. \latextohtml{} requires a \texttt{bbl}
in order to generate the references.

\item [The labels of figures, tables or equations are wrong]
This can happen if you have used any figures, tables, equations or
any counters inside conditional text i.e. in a \texttt{latexonly} 
or a \texttt{htmlonly} environment. 

\item [Problems after changing the configuration files]
Please make sure that the last line in the configuration files 
(ie \fn{.latex2html-init} and \fn{latex2html.congif}) is:
\begin{verbatim}
1;	# This is the last line
\end{verbatim}
This is a Perl quirk...

\item [Problems when producing the DVI version \label{htmlsty}]
If you are using any of the new LaTeX commands which are defined in 
the \fn{html.sty} file make sure that 
\fn{html.sty} file is included e.g. as one of the optional arguments to the 
\texttt{documentstyle} command.

Of course you also have to make sure that LaTeX knows where the html.sty
file is, either by putting it in the same place as the other style files on
your system, or by changing your TEXINPUTS shell environment variable\footnote{
If don't know how to do either of these things, copy (or link) html.sty 
to the directory of your LaTeX document...}.

\item [Some of the fonts are translated incorrectly]
There is a fault in way the LaTeX scoping rules have been 
interpreted in \latextohtml. Consider this:
\begin{verbatim}
\ttfamily fixed-width font.
\begin{something}
nothing here
\end{something}
default font.
\end{verbatim}
When processed by \LaTeX, the effect of the \texttt{tt} command is
delimited
by the beginning of the environment ``something'' so that ``default font'' will
appear in the default font. But \latextohtml{} will not recognize
``something'' as a delimiter and ``default font'' will appear in the
wrong
font. 

To avoid this problem until it is fixed you may delimit the scope of
some
commands explicitly using \verb|{}|'s i.e.
\begin{verbatim}
\texttt{fixed-width font}.
\begin{something}
nothing here
\end{something}
default font.
\end{verbatim}

\item [Using \fn{Ghostscript 3.X} you can no
longer generate inlined images for equations]
If you have a version of \latextohtml{} later than 0.6.1, go to the
\latextohtml{} directory and run \fn{install-test} again. This should 
fix it.

With earlier versions of \latextohtml{} you can fix it by 
replacing the file \fn{pstoppm.ps} in the 
\latextohtml{} directory with a newer one that accompanies 
\fn{Ghostscript 3.X}. Alterhatively you can avoid using {\fn
pstoppm.ps} 
by changing the way \texttt{GS} is invoked in the file \fn{pstogif},
using something like \\
\verb/open (GS, "|$GS -q -sDEVICE=ppmraw  -sOutputFile=$base.ppm $base.ps");/

\item [Cannot get it to generate inlined images]
Try a small test file e.g.
\begin{verbatim}
% image-test.tex
\documentstyle{article}
\begin{document}
Some text followed by \fbox{some more text in a box}.
\end{document}
\end{verbatim}

You should see something like:
\begin{verbatim}
This is LaTeX2HTML Version  (Wed Dec 1 1993) by Nikos Drakos, 
Computer Based Learning Unit, University of Leeds.

OPENING /usr/cblelca/nikos/scripts/tex2html/tests/image-test.tex 

Reading ....
Translating ...0/1.....1/1......
Generating images using latex ...
This is TeX, C Version 3.14t3
12222_images.tex
LaTeX Version 2.09 <7 Dec 1989>


Generating postscript images using dvips ...
This is dvips 5.521 Copyright 1986, 1993 Radical Eye Software
\end{verbatim}
\begin{verbatim}
' TeX output 1993.12.03:1050' -> 12222_image
(-> 12222_image001) <tex.pro>[1] 
Initializing... done.
Ghostscript 2.6.1 (5/28/93)
Copyright (C) 1990-1993 Aladdin Enterprises, Menlo Park, CA.
  All rights reserved.
Ghostscript comes with NO WARRANTY: see the file COPYING for details.
GS>GS>Writing 12222_image001.ppm
GS>pnmcrop: cropping 119 rows off the top
pnmcrop: cropping 961 rows off the bottom
pnmcrop: cropping 208 cols off the left
pnmcrop: cropping 484 cols off the right

Doing section links .....
Done.
\end{verbatim}


If there is a problem somewhere during the conversion from postscript
to GIF you can try to do it manually so that you can find out where
the problem is. Here is one way to do it (Please use the \fn{pstoppm3.ps}
file instead of \fn{pstoppm.ps} if your version of ghostscript is
later than 3.0):

\begin{verbatim}
cblelca% latex image-test.tex
This is TeX, C Version 3.14t3
(image-test.tex
LaTeX Version 2.09 <7 Dec 1989>
(/usr/TeX/tex.lib/inputs//paper.sty
Document Style `paper' <28 Nov 89>.
(/usr/TeX/tex.lib/inputs//pap11.sty) (/usr/TeX/tex.lib/inputs/\-doublespace.sty)
(/usr/TeX/tex.lib/inputs//smaller.sty)) (/usr/TeX/tex.lib/inputs\-/psfig.sty
psfig/tex 1.9
)
No file image-test.aux.
[1] (image-test.aux) )
Output written on image-test.dvi (1 page, 652 bytes).
Transcript written on image-test.log.
cblelca% dvips -o image-test.ps image-test.dvi
\end{verbatim}
\begin{verbatim}
This is dvips 5.519 Copyright 1986, 1993 Radical Eye Software
' TeX output 1993.11.12:1412' -> image-test.ps
<tex.pro>. [1] 
cblelca% gs -dNODISPLAY pstoppm.ps
Initializing... done.
Ghostscript 2.6.1 (5/28/93)
Copyright (C) 1990-1993 Aladdin Enterprises, Menlo Park, CA.
  All rights reserved.
Ghostscript comes with NO WARRANTY: see the file COPYING for details.
GS>(image-test) ppm1run 
Writing image-test.ppm
GS>quit
cblelca% pnmcrop image-test.ppm >image-test.crop.ppm
pnmcrop: cropping 61 rows off the top
pnmcrop: cropping 110 rows off the bottom
pnmcrop: cropping 72 cols off the left
pnmcrop: cropping 72 cols off the right
cblelca% ppmtogif image-test.crop.ppm >image-test.gif
\end{verbatim}

\item [STILL cannot get it to generate inlined images for equations
etc.]
If you have no problems with the \fn{image-test.tex} file but you
still cannot convert the images in some of your files 
have a look in the directory of the generated
HTML files for two files \fn{images.tex} and \fn{images.log}. Do you notice
anything unusual in them? Copy \fn{images.tex} in the directory 
of your original \LaTeX file and run \fn{latex} on \fn{images.tex}.
Can you see any errors in \fn{images.log}? If yes can you fix
\fn{images.tex} to get rid of the errors? After fixing {\fn
images.tex}
you can put it back in the directory of HTML files created by
\latextohtml{} and run \latextohtml{} on the original document 
using the option \texttt{-images\_only}. 

If you get into a mess try running \latextohtml{} with the options
\texttt{-no\_reuse} and \texttt{-no\_images} eg
\begin{verbatim}
cblipca% latex2html -no_reuse -no_images test.tex
This is LaTeX2HTML Version 95 (Tue Nov 29 1994) by Nikos Drakos, 
Computer Based Learning Unit, University of Leeds.

OPENING /tmp_mnt/home/cblelca/nikos/tmp/test.tex 
Cannot create directory /usr/cblelca/nikos/tmp/test: File exists
(r) Reuse the images in the old directory OR
(d) *** DELETE *** /usr/cblelca/nikos/tmp/test AND ITS CONTENTS OR
(q) Quit ?
:d

Reading ...
Processing macros ....+.
Reading test.aux ......................
Translating ...0/1........1/1.....
Writing image file ...

Doing section links .....

*********** WARNINGS ***********

If you are having problems displaying the correct images with Mosaic,
try selecting "Flush Image Cache" from "Options" in the menu-bar 
and then reload the HTML file.

Done.
\end{verbatim}

Then try to have a look 
in the file  \fn{images.tex} (as described earlier) and perhaps fix it.
Once you are happy that \fn{images.tex} is OK run \latextohtml{}
again with the option \texttt{-images\_only}.

The options \texttt{no\_reuse, no\_images} and \texttt{images\_only}
are available with \latextohtml{} version 0.7 or later.

Some problems in displaying the correct inlined images,
may be due to the image caching mechanisms of your browser.
With some browsers a simple ``Reload Current Document'' will be enough
to refresh the images but with others (eg Mosaic) you may need
to request for the cache to be refreshed. With Mosaic try 
selecting "Flush Image Cache" from "Options" in the menu-bar 
and then reload the HTML file.


\item [It cannot do slides, memos, etc, ...] 
If you use \texttt{slitex} you can go a long way just by replacing 
the \texttt{slides} argument of the \texttt{documentstyle} command with 
something like \texttt{article} just before using \latextohtml.
One problem may be that all your slides will end up in the  same HTML 
file.
If you use \fn{lslide.sty} you may get much better results 
(\htmladdnormallinkfoot{use Archie}
{http://www.pvv.unit.no/archie/} to find this or any other
style files).
\end{htmllist}


\section{Support and More Information}

A \htmladdnormallink{\latextohtml{} mailing list}{mailto:latex2html-request@mcs.anl.gov} has been set up at the
Argonne National Labs (thanks to Ian Foster 
\Email{itf@mcs.anl.gov} and Bob Olson \Email{olson@mcs.anl.gov}). The
\htmladdnormallinkfoot{\latextohtml{} mailing list
archive}{http://cbl.leeds.ac.uk/nikos/tex2html/doc/mail/mail.html} is
available.

To join send a message to: \\
\texttt{latex2html-request@mcs.anl.gov}  \\
with the contents \\
\texttt{subscribe}

To be removed from the list send a message to: \\
\texttt{latex2html-request@mcs.anl.gov}  \\
with the contents \\
\texttt{unsubscribe}.

\section{General License Agreement and Lack of Warranty}
This software is distributed in the hope that it will be useful
but \textbf{without any warranty}. The author(s) do not accept responsibility 
to anyone for the consequences of using it or for whether it serves 
any particular purpose or works at all. No warranty is made about 
the software or its performance. 
 
Use and copying of this software and the preparation of derivative
works based on this software are permitted, so long as the following
conditions are met:
\begin{itemize}
\item The copyright notice and this entire notice are included intact
and prominently carried on all copies and supporting documentation.
\item No fees or compensation are charged for use, copies, or
access to this software. You may charge a nominal
distribution fee for the physical act of transferring a
copy, but you may not charge for the program itself. 
\item If you modify this software, you must cause the modified
file(s) to carry prominent notices (a Change Log)
describing the changes, who made the changes, and the date
of those changes.
\item  Any work distributed or published that in whole or in part
contains or is a derivative of this software or any part 
thereof is subject to the terms of this agreement. The 
aggregation of another unrelated program with this software
or its derivative on a volume of storage or distribution
medium does not bring the other program under the scope
of these terms.
\end{itemize}
 
This software is made available \textbf{as is}, and is distributed without 
warranty of any kind, either expressed or implied.
In no event will the author(s) or their institutions be liable to you
for damages, including lost profits, lost monies, or other special,
incidental or consequential damages arising out of or in connection
with the use or inability to use (including but not limited to loss of
data or data being rendered inaccurate or losses sustained by third
parties or a failure of the program to operate as documented) the 
program, even if you have been advised of the possibility of such
damages, or for any claim by any other party, whether in an action of
contract, negligence, or other tortious action.

\index{copyright}
The \latextohtml{} translator is written by Nikos Drakos, 
Computer Based Learning Unit,  University of Leeds,  Leeds,  LS2 9JT.
Copyright \copyright 1993, 1994, 1995. All rights reserved.
 
\section{Credits}
Several people have contributed suggestions, ideas, solutions, support
and encouragement. Some of these are Roderick Williams, Ana Maria
Paiva, Jamil Sawar and Andrew Cole here at the Computer Based Learning Unit.

The idea of splitting \LaTeX\  files
into more than one components linked with hyperlinks was first
implemented in Perl by Toni Lantunen at CERN.
Thanks to Robert Cailliau \Email{cailliau@cernnext.cern.ch}
of the WorldWide Web Project also at CERN 
for giving me access to the source code and documentation (although no
part of the original design or the actual code has been used).

Robert S. Thau \Email{rst@edu.mit.ai} has contributed the new version of
\fn{texexpand}. Also, in order to translate the \emph{document 
from hell} (!!!) he has extended the translator to handle \texttt{def}
commands, nested math-mode commands, and has fixed several bugs.

The \fn{pstogif} script 
uses the \fn{pstoppm.ps} postscript program originally written by
Phillip Conrad (Perfect Byte, Inc.) and modified by L. Peter Deutsch 
(Aladdin Enterprises).

The idea of using existing symbolic labels to provide cross-references
between documents was first conceived during discussions with 
Roderick Williams \Email{rodw@cbl.leeds.ac.uk}.
Eric Carroll \Email{eric@ca.utoronto.utcc.enfm} suggested providing
a command like \texttt{hyperref}. Franz Vojik
\Email{vojik@de.tu-muenchen.informatik}
provided the basic mechanism for handling foreign accents.
The \texttt{-auto\_navigation} option was based on an idea by
Todd \texttt{<little@com.dec.enet.nuts2u>}.
Axel Belinfante \texttt{<Axel.Belinfante@cs.utwente.nl>} provided the
code in the \fn{makeidx.perl} file as well as numerous suggestions
and bug reports. 

Verena Umar \Email{verena@edu.vanderbilt.cas.compsci} 
(\htmladdnormallink{Computer Science
Education Project}{http://csep1.phy.ornl.gov/csep.html}) has been a very
patient tester of some early versions of \latextohtml{}
and many of the current features are a result of her 
suggestions. 

Thanks to (thanks to Ian Foster 
\Email{itf@mcs.anl.gov} and Bob Olson \Email{olson@mcs.anl.gov}) at the
Argonne National Labs for setting up the
\htmladdnormallinkfoot{\latextohtml{} mailing list}{http://cbl.leeds.ac.uk/nikos/tex2html/doc/mail/mail.html}.

\begin{changebar}
Special credit is due Marcus Hennecke \Email{marcush@crc.ricoh.com}
for his many extensive revisions, Mark Noworolski
\Email{jmn@eecs.berkeley.edu} for coordinating V 95.3, 
Sidik Isani \Email{isani@cfht.hawaii.edu}, for his improvement
in GIF quality, and to Herb Swan \Email{dprhws@edp.Arco.com} for
coordinating V 96.1 of \latextohtml.
\end{changebar}

Many others, too many to mention, 
have contributed bug reports, fixes, and other suggestions. Keep them coming!

%\end{htmlonly}

\endsegment[section]

%%% START XTRACTFAQ
%
%  BIBLIOGRAPHY
%
\bibliographystyle{plain}
\newcommand{\AWcsengURL}{http://www.awl.com/cseng}
\newcommand{\AddWes}[1]{\htmladdnormallink{Addison--Wesley}{#1}}
\newcommand{\awlogo}{\AWcsengURL/images/tri.gif}
\newcommand{\AWlamport}{\AWcsengURL/titles/0-201-52983-1}
\newcommand{\LAMPcover}{\AWlamport/coversm.gif}
\newcommand{\AWtheTLC}{\AWcsengURL/titles/0-201-54199-8}
\newcommand{\TLCcover}{\AWtheTLC/tlc.gif}
\newcommand{\AWtheLGC}{\AWcsengURL/titles/0-201-85469-4}
\newcommand{\LGCcover}{\AWtheLGC/coversm.gif}
\newcommand{\AWtheLWC}{\AWcsengURL}
\newcommand{\LWCcover}{}
\begin{thebibliography}{1}\label{biblio}

\index{LaTeX blue book@\LaTeX{} blue book!Leslie Lamport}%
\htmladdimg[ALIGN=RIGHT width=80 height=100]{\LAMPcover}
\bibitem{lamp:latex}
Leslie Lamport,
\newblock \LaTeX,\textit{A Document Preparation System}.
 User's Guide \& Reference Manual, 2nd edition.
\newblock ISBN 0-201-52983-1, Paperback 256 pages, 
 \AddWes{\AWlamport}, 1994.
\newblock Online information on {\TeX} and {\LaTeX} is available at \tugURL\ and~\danteURL~.

%begin{latexonly}
\index{Companion|see{The \LaTeX\hfil Companion}}%
\index{LaTeX Companion@\LaTeX\ Companion|see{~The \LaTeX\\ Companion}}%
\index{The LaTeX Companion@\emph{The {\upshape\LaTeX} Companion}\label{IIIlatcomp}!Goossens--Mittelbach--Samarin}%
%end{latexonly}
\begin{htmlonly}
\index{Companion|see{\htmlref{~The \textup{\LaTeX} Companion}{IIIlatcomp}}}%
\index{LaTeX Companion@\LaTeX\ Companion|see{\htmlref{The \textup{\LaTeX} Companion}{IIIlatcomp}}}%
\index{The LaTeX Companion@The \textup{\LaTeX} Companion\label{IIIlatcomp}!Goossens--Mittelbach--Samarin}%
\end{htmlonly}
\htmladdimg[ALIGN=RIGHT,WIDTH=61,HEIGHT=76]{\TLCcover}
\bibitem{goossens:latex}
Michel Goossens, Frank Mittelbach, Alexander Samarin,
\newblock \emph{The }\LaTeX\emph{ Companion}
\newblock ISBN 0-201-54199-8, Paperback 530 pages, \AddWes{\AWtheTLC/}, 1994.
\html{\bigskip\bigskip}

%begin{latexonly}
\index{Graphics Companion|see{~The~\textup{\LaTeX}\\ Graphics Companion}}%
\index{LaTeX Graphics Companion@\LaTeX\ Graphics Companion|see{~The\\ \textup{\LaTeX} Graphics Companion}}%
\index{The LaTeX Graphics Companion@\emph{The \textup{\LaTeX} Graphics Companion}\label{IIIlatgraph}!Goossens--Rahtz--Mittelbach}%
%end{latexonly}
\begin{htmlonly}
\index{Graphics Companion|see{\htmlref{The \textup{\LaTeX}\latex{\hfil} Graphics Companion}{IIIlatgraph}}}%
\index{LaTeX Graphics Companion@\LaTeX{} Graphics Companion|see{\htmlref{The \textup{\LaTeX}\latex{\hfil} Graphics Companion}{IIIlatgraph}}}%
\index{The LaTeX Graphics Companion@The \textup{\LaTeX} Graphics Companion\label{IIIlatgraph}!Goossens--Rahtz--Mittelbach}%
\end{htmlonly}
\htmladdimg[ALIGN=RIGHT]{\LGCcover}
\bibitem{goossens:latexGraphics}
Michel Goossens, Sebastian Rahtz and Frank Mittelbach,
\newblock \emph{The }\LaTeX\emph{ Graphics Companion}.
\newblock ISBN 0-201-85469-4, Softcover 608 pages, \AddWes{\AWtheLGC/}, 1997.

%begin{latexonly}
\index{Web Companion|see{~The~\textup{\LaTeX}\\ Web Companion}}%
\index{LaTeX Web Companion@\LaTeX\ Web Companion|see{~The\\ \textup{\LaTeX} Web Companion}}%
\index{The LaTeX Web Companion@\emph{The \textup{\LaTeX} Web Companion}\label{IIIlatweb}!Rahtz--Goossens et al.}%
%end{latexonly}
\begin{htmlonly}
\index{Graphics Web|see{\htmlref{The \textup{\LaTeX}\latex{\hfil} Web Companion}{IIIlatweb}}}%
\index{LaTeX Web Companion@\LaTeX{} Web Companion|see{\htmlref{The \textup{\LaTeX}\latex{\hfil} Web Companion}{IIIlatweb}}}%
\index{The LaTeX Web Companion@The \textup{\LaTeX} Web Companion\label{IIIlatweb}!Rahtz--Goossens et al.}%
\end{htmlonly}
%\htmladdimg[ALIGN=RIGHT]{\LWCcover}
\bibitem{rahtz:latexWeb}
Sebastian Rahtz and Michel Goossens, with E. Gurari, R. Moore \& R. Sutor.
\newblock \emph{The }\LaTeX\emph{ Web Companion}.
\newblock to appear in 1999, \AddWes{\AWtheLWC/}.

\bibitem{drakos:bask}
\NikosDrakos,
\newblock Text to Hypertext conversion with \latextohtml.
\newblock \textit{Baskerville}, December 1993, Vol.\,3, No.\,2, pp 12--15.
\newblock \url{http://cbl.leeds.ac.uk/nikos/doc/www94/www94.html}

\bibitem{drakos:www94}
\NikosDrakos,
\newblock From Text to Hypertext: A Post-Hoc Rationalisation of \latextohtml.
\newblock Published in ``Proceedings of the 1st World Wide Web Conference'',
\newblock May 1994, CERN, Geneva, Switzerland.
\newblock \url{http://cbl.leeds.ac.uk/nikos/doc/www94/www94.html}

\end{thebibliography}
%%% END XTRACTFAQ

%\end{document}

%
%  GLOSSARY
%
% Glossary info stored in:  manual.gls ,  which was created using:
%
%       makeindex -o manual.gls -s l2hglo.ist manual.glo
%       
\begin{latexonly}
\InputIfFileExists{manual.gls}{\clearpage\typeout{^^Jcreating Glossary...}}%
{\typeout{^^JNo Glossary, since  manual.gls  could not be found.^^J}}
\end{latexonly}

\begin{htmlonly}
\section{Glossary of variables and file-names\label{Glossary}}
\begin{htmllist}\htmlitemmark{OrangeBall}
\input l2hfiles.dat
\end{htmllist}
\end{htmlonly}

%
%  INDEX
%
\internal[index]{O}
\internal[index]{S}
\internal[index]{E}
\internal[index]{H}
\internal[index]{M}
\internal[index]{P}
%
% Index info stored in:  manual.ind ,  which was created using:
%
%       makeindex -s l2hidx.ist manual.idx
%
\printindex

%
%  Alphabetization and navigation within the index
%  ...these special index entries must come *after* the  \printindex
%  else half of the hyperlinks will point to the preceding page.
%
\begin{htmlonly}
\newcommand{\indexAlpha}[5]{\index{#1@\htmlref{_}{#2}%
 \htmlref{\HTML{SUB}{\LARGE #3}}{AZ}\htmlref{_}{#4}\label{#5}| }}
%
\indexAlpha{\$}{Z}{\$}{dot}{doll}%
\indexAlpha{.}{doll}{~.~}{A}{dot}%
\indexAlpha{A}{dot}{A}{B}{A}%
\indexAlpha{B}{A}{B}{C}{B}%
\indexAlpha{C}{B}{C}{D}{C}%
\indexAlpha{D}{C}{D}{E}{D}%
\indexAlpha{E}{D}{E}{F}{E}%
\indexAlpha{F}{E}{F}{G}{F}%
\indexAlpha{G}{F}{G}{H}{G}%
\indexAlpha{H}{G}{H}{I}{H}%
%\indexAlpha{I}{H}{I}{J}{I}%
\indexAlpha{I}{H}{I}{L}{I}%
\indexAlpha{J}{I}{J, K}{L}{K}%
%\indexAlpha{J}{I}{J}{K}{J}%
%\indexAlpha{K}{J}{K}{L}{K}%
\indexAlpha{L}{K}{L}{M}{L}%
\indexAlpha{M}{L}{M}{N}{M}%
\indexAlpha{N}{M}{N}{O}{N}%
\indexAlpha{O}{N}{O}{P}{O}%
%\indexAlpha{P}{O}{P}{Q}{P}%
\indexAlpha{P}{O}{P}{R}{P}%
\indexAlpha{Q}{P}{Q, R}{S}{R}%
%\indexAlpha{Q}{P}{Q}{R}{Q}%
%\indexAlpha{R}{Q}{R}{S}{R}%
\indexAlpha{S}{R}{S}{T}{S}%
\indexAlpha{T}{S}{T}{U}{T}%
\indexAlpha{U}{T}{U}{V}{U}%
\indexAlpha{V}{U}{V}{W}{V}%
\indexAlpha{W}{V}{W}{X}{W}%
%\indexAlpha{X}{W}{X}{Y}{X}%
\indexAlpha{X}{W}{X}{Z}{X}%
\indexAlpha{Y}{X}{Y, Z}{doll}{Z}%
%\indexAlpha{Y}{X}{Y}{Z}{Y}%
%\indexAlpha{Z}{Y}{Z}{doll}{Z}%
%
%
%% This is an alphabetical navigation panel.
\index{@\label{AZ}\textbf{\LARGE
\htmlref{\$}{doll} \htmlref{.}{dot} \htmlref{A}{A} 
 \htmlref{B}{B} \htmlref{C}{C} \htmlref{D}{D} \htmlref{E}{E} \htmlref{F}{F}
 \htmlref{G}{G} \htmlref{H}{H} \htmlref{I}{I} \htmlref{J}{K} \htmlref{K}{K}
 \htmlref{L}{L} \htmlref{M}{M} \htmlref{N}{N} \htmlref{O}{O} \htmlref{P}{P}
 \htmlref{Q}{R} \htmlref{R}{R} \htmlref{S}{S} \htmlref{T}{T} \htmlref{U}{U}
 \htmlref{V}{V} \htmlref{W}{W} \htmlref{X}{X} \htmlref{Y}{Z} \htmlref{Z}{Z}}\\
\htmlrule[all]| }

\end{htmlonly}

\end{document}
