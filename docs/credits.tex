\section*{Credits, 1993--1994\label{credits}}%
\index{Computer~Based~Learning~Unit!University of Leeds}\html{\\}%
Several people have contributed suggestions, ideas, solutions, support
and encouragement. Some of these are \RodWilliams, \AnaPaiva, 
\JamilSawar\ and \AndrewCole\ at the \CBLU.

\begin{htmllist}
\htmlitemmark{YellowBall}
\index{CERN!World-Wide Web Project}%
\item [\CERN]
The idea of splitting \LaTeX{}  files
into more than one component, connected via hyperlinks, 
was first implemented in \Perl{} by Toni Lantunen at CERN.
Thanks to \RobertCailliau{} of the World-Wide Web Project, also at CERN, 
for providing access to the source code and documentation 
(although no part of the original design or the actual code has been used).


\item [\RobertThau]
has contributed the new version of \fn{texexpand}. 
Also, in order to translate the ``document from hell'' (!!!) 
he has extended the translator to handle \Lc{def} commands, 
nested math-mode commands, and has fixed several bugs.

\item [Phillip Conrad and L. Peter Deutsch.]
The \fn{pstogif} \Perl{} script uses the \fn{pstoppm.ps} \PS\ program, 
originally written by Phillip Conrad (Perfect Byte, Inc.) and 
modified by L. Peter Deutsch (Aladdin Enterprises).

\item [\RodWilliams]
The idea of using existing symbolic labels to provide cross-references
between documents was first conceived during discussions with Roderick.


\item [\EricCarroll] 
who first suggested providing a command like \Lc{hyperref}\,. 

\index{accents!foreign}%

\item [\FranzVojik]
provided the basic mechanism for handling foreign accents.


\item [\ToddLittle]
The \Cs{auto\_navigation} option was based on an idea by Todd.


\item [\AxelBelinfante]
provided the \Perl{} code in the \fn{makeidx.perl} file, 
as well as numerous suggestions and bug-reports. 

\item [\VerenaUmar] (from the \CSEP) 
has been a very patient tester of some early versions of \latextohtml{}
and many of the current features are a result of her suggestions. 

\index{mailing list!Argonne National Labs}%

\item [Ian Foster and Bob Olson.]
Thanks to \IanFoster{} and \BobOlson{} at the Argonne National Labs, 
for setting up the \maillist.
\end{htmllist}


\clearpage
\section*{Later Developments, 1995--1996\label{recent96}}%
%
Since 1995 the power and usefulness of \latextohtml{} has been enhanced significantly.
The revisions later than \textsc{v95.1} have been largely due 
to the combined efforts of many people, other than the original author.
Interested users have supplied patches to fix a fault, 
or implement a feature that previously was not supported. 
Often a question or complaint to the discussion-group 
(see \hyperref{Getting Support ...}{Section~}{}{support})
has spurred someone else to provide the necessary ``patch''.%

\bigskip\noindent
Arising from this work, special credit is due to: 
\begin{htmllist}
\htmlitemmark{GreenBall}
\item [\Hennecke] for his many extensive revisions; 

\item [\Noworolski] for coordinating \textsc{v95.3};
 
\item [\Isani] for his improvement in GIF quality; 

\index{CERN!World-Wide Web Project}\index{CERN!Michel Goossens}%
\item [\Goossens]
was the driving force behind the upgrade to \LaTeXe{} compatibility,
and other features developed at CERN;

\item [Herb Swan]
for coordinating \textsc{v96.1} of \latextohtml, 
including much of the \Perl{} code 
for the new features that were introduced,
and for providing a series of bug-fix revisions 
prior to  \textsc{v96.1} \texttt{rev-f};

\item [\RossMoore]
who has revised and extended this manual, helped design and test the 
segmentation strategy, and later revisions of \textsc{v96.1}\,.
Ross organised the release of \textsc{v96.1} \texttt{rev-g} 
and provided many of the improvements
incorporated into \textsc{v96.1} \texttt{rev-h}. 

\item [\Wilck]
for the initial work on implementation of \env{frames}. 
Also Martin did most of the work implementing the extensive citation and
bibliographic features of the \env{natbib} package, written by \PatrickDaly.
He also provided the \fn{makeseg} \Perl{} script to create Makefiles
for segmented documents.

\item [\Lippmann]
for organising the releases \textsc{v96.1} \texttt{rev-h} to \textsc{v98.1}. 
Jens made significant contributions to
the internal workings of \latextohtml, 
as well as cleaning up much of its source code. 
\end{htmllist}

\htmlrule


\bigskip\noindent
Many others, too many to mention, contributed bug-reports, 
fixes and other suggestions.

\latex{\bigskip}\htmlrule

\index{URL!url package@\env{url} package}%
\noindent
Thanks also to \Arseneau{} for allowing his \fn{url.sty} 
to be distributed with this manual. 
Similarly, thanks to \JohannesBraams{} for \fn{changebar.sty}.
%
\index{change-bars}\html{\\}
Both of these are useful utilities which enhance the appearance of the printed manual. 
\begin{latexonly}
In particular, changes introduced with  \textsc{v98.1} and its revisions are denoted 
by thin change-bars, while thicker bars denote changes introduced with  \textsc{v98.2}
and later releases.%
\end{latexonly}



\clearpage
\section*{Developments: late 1996 to mid 1997\label{recent97}}%
%
During the latter part of 1996 there was much work on improving the
capabilities of \latextohtml.
Some of this was due to the \WiiiC's proposals for \HTMLiii, 
becoming a formal recommendation in November 1996,
and their subsequent acceptance in January 1997. 
Existing \LaTeX{} markup for effects such as centering, left- 
or right-justification of paragraphs,
flow of text around images, table-layout with formal captions, etc.
could now be given a safe translation into \HTMLiii, compliant with a standard
that would guarantee that browsers would be available to view such effects.  

At the same time developers were exploring ways to enhance the overall
performance of \latextohtml.
As a result the current \textsc{v97.1} release has significant improvements in
the following areas:
%
\begin{htmllist}\htmlitemmark{OrangeBall}
% 
\item[image-generation]
is much faster, requires less memory
and inline images are aligned more accurately; 
%
\item[image quality]
is greatly improved by the use of anti-aliasing effects for on-screen clarity,
in particular with mathematics, text and line-drawings; 
%
\item[memory-requirements]
are much reduced, particularly with image-generation;
%
\item[mathematics]
can now be handled using a separate parsing procedure;
images of sub-parts of expressions can be created, rather
than using a single image for the whole formula;
%
\item[macro definitions]
having a more complicated structure than previously allowed,
can now be successfully expanded;
%
\item[counters]
and numbering are no longer entirely dependent on the \texttt{.aux}
file generated by \LaTeX;
%
\item[decisions]
about which environments to include or exclude can now be made;
%
\item[HTML effects]
for which there is no direct \LaTeX{} counterpart
can be requested in a variety of new ways;
%
\item[HTML code]
produced by the translator is much neater and more easily readable,
containing more comments and fewer redundant breaks and \HTMLtag{P} tags.
%
\item[error-detection]
of simple \LaTeX{} errors, such as missing or unmatched braces, 
is now performed --- a warning message shows a line or two
of the source code where the error has apparently occurred;
%
\end{htmllist} 


\medskip\htmlrule[50\% center]
\noindent
For these developments, thanks goes especially to: 
%
\begin{htmllist}
%
\item [\Lippmann]
for creating and maintaining the CVS repository at \CVSrepos\,. 
This has made it much easier for the contributions from different developers
to be collected and maintained as a ``development version'' which
is kept up-to-date and available at all times. Together with \Rouchal\
he produced an extensive rewrite of the \fn{texexpand} utility.


\item [\RossMoore]
for extensive work on almost all aspects of the \latextohtml{} source
and documentation,
combining code for \LaTeX{}, \Perl{}, \texttt{HTML} and other utilities.
Most of the coding for the new features based on \texttt{HTML} 3.2, 
many of the new packages, faster image-generation 
and the improved support for mathematics 
and other environments, is his work.


\item [\Rouchal]
for extending the former \fn{pstogif} utility,
transforming it into \fn{pstoimg} which now allows for 
alternative image formats, such as \fn{PNG}.
Also he produced the neat \fn{configure-pstoimg} script, which eases
\latextohtml{} installation, and a rewrite of \fn{texexpand}.



\index{portability!Unix systems}\index{latex2html-NG}%
\item [\Hennecke]
who has always been there, up-to-date with developments in \texttt{HTML} and
related matters concerning Web publishing, 
and tackling the issues involved with portability 
of \latextohtml{} to Unix systems on various platforms.

Furthermore Marcus has produced \latextohtmlNG, a version of
\latextohtml{} which handles expansion of macros in a more ``\TeX-like''
fashion. This should lead to further improvements in speed and efficiency,
while allowing complicated macro definitions to work as would be expected
from their expansions under \LaTeX.
(This requires \Perl{ 5}\,, 
using some programming features not available with \Perl{ 4}\,.)%


\item [\Popineau] has produced
an adaptation for the Windows NT platform, of \latextohtml{} \textsc{v97.1}\,.

\item [\Wortmann]
showed how to configure \appl{Ghostscript} to produce
anti-aliasing effects within images.

\item [\AxelRamge]
for various suggestions and examples of enhancements,
and the code to avoid a problem with \appl{Ghostscript}.

\end{htmllist}

\medskip\vfil\htmlrule\bigskip\noindent
Thanks also to all those who have made bug-reports, supplied fixes 
or offered suggestions as to features that might allow \latextohtml{} 
to be used more efficiently in particular circumstances. 
Most of these have been incorporated into this new version \textsc{v97.1}\,,
though perhaps not in the form originally envisaged.
Such feedback has contributed enormously to helping make \latextohtml{} the
easy to use, versatile program that it has now become.

\bigskip
\begin{center}
Keep the ideas coming!
\end{center}
\bigskip
\vfil




\subsection*[center]{1st \LaTeX2HTML{} Workshop\\Darmstadt, 
15 February 1997\label{darmstadt}}
Thanks again to \Lippmann\ and members of the \LiPS\ for organising this meeting; 
also to the \FIDarmstadt\ at \Darmstadt\ for use of their facilities.

\noindent
This was an opportunity for many of the current \latextohtml{} developers to
actually meet for the first time; rather than communication by exchange
of electronic mail messages.
%
\begin{itemize}
\item
\NikosDrakos\ talked about the early development of \latextohtml, while\dots
\item \dots
\RossMoore, \Lippmann\ and \Rouchal\ described recent improvements. 
\item
\Goossens\ presented a list of difficulties encountered with earlier
versions of \latextohtml{}, and aspects requiring improvement.
Almost all of these now have been addressed in the \textsc{v97.1} release,
so far as is possible within the bounds inherent in the \HTMLiii\ standard.
\item
\KrisRose\ showed how it is possible to create \texttt{GIF89} animations 
from pictures generated by \TeX{} or \LaTeX{}, using the \XypicDK\ graphics 
package and extensions, developed by himself and \XypicAUS.
\end{itemize}

\noindent
Also present were representatives from the \DANTE\ \Praesidium\ and 
members of the \LaTeXiii\ development team.\html{\\} 
In all it was a very pleasant and constructive meeting.

\vfil

\subsection*[center]{TUG'97 --- Workshop on \LaTeX2HTML\\
University of San Francisco, 28 July 1997\label{tug97}}

\noindent
On the Sunday afternoon (2.00pm--5.00pm)
immediately prior to the TUG meeting, there will be a workshop
on \latextohtml, conducted by \RossMoore\footnote{%
Mathematics Department, Macquarie University, Sydney, Australia}.

\begin{itemize}
\item[]
Admission: \$50, includes a printed copy of the latest \latextohtml\ manual.
\end{itemize}

\bigskip

\subsection*[center]{\TeX{}Northeast TUG Conference, \TeX/\LaTeX{} Now\\
March 22--24, 1998, New York City}

\noindent
Includes a workshop/presentation by \RossMoore\footnote{%
Mathematics Department, Macquarie University, Sydney, Australia}.


\subsection*[center]{Euro-\TeX{}'98, 10th European \TeX{} Conference\\
St. Malo, France --- 29--31 March, 1998}

\noindent
Several of the \latextohtml{} developers will be present.
All European (and other) \latextohtml{} users are encouraged to attend.


\vfil


\clearpage
\section*{Developments: late 1997 to early 1998\label{recent98}}%
Much of the work contributed to \latextohtml{} during this time was
related to bug fixing and maintaining the 97.1 release, in order to
reach a more stable and reliable version which produces \fn{HTML} code
conforming to the W3C standards/drafts.
To keep in context with this view, support for \texttt{HTML 4} has been
incorporated into the translator.

\smallskip\noindent
There have been improvements to the way math code is handled, as well
as font-changing and numbering commands. These now are expected to
work much closer to the way that \LaTeX{} handles them.

\smallskip\noindent
Furthermore, missing \LaTeX{} style translations for basic \LaTeX{} 
and \AmSTeX{} document classes were added to the distribution: 
\texttt{book.perl}, \texttt{report.perl}, \texttt{article.perl},
\texttt{letter.perl}, \texttt{amsbook.perl} and \texttt{amsart.perl}.
New styles implementing \LaTeX{} packages include \texttt{seminar.perl},
\texttt{inputenc.perl} and \texttt{chemsym.perl} naming but a few.

The aim is ultimately to support all \LaTeX{}, \AmSTeX{} etc. packages in the 
standard \LaTeX{} distribution, or for which there is published documentation. 
At the time of writing this aim has not quite been reached.
To support internationalisation, \Perl{} extensions were provided for
\texttt{HTML} output conforming to ISO-Latin 1, 2, 3, 4, 5, 6,
and \Unicode{} encodings.

\smallskip\noindent
All of the above work was done by \RossMoore.
\bigskip

Additional document formats are now supported, these are Indic\TeX{},
\FoilTeX{}, and \texttt{CWEB} documents.
You may use any of these packages to translate such documents together
with \latextohtml{}, refer to the instructions in the various
\texttt{README} files.

\smallskip\noindent
Thanks go to \RossMoore{} for \IndicHTML{}, to \Veytsman{} for 
\FoilTeX{}/\texttt{HTML} and to \Lippmann{} for the \texttt{CWEB} to
\texttt{HTML} translator.

\bigskip
Numerous discussions and efforts have been undertaken to get
\latextohtml{} working independent from the underlying operating
system.
Yet all obstacles are not quite taken, but it is forseeable that we are
OS independent very soon.
This release has been reported to run on OS/2, DOS, and MacOS, besides
Unix-like operating systems.
A former version has also been ported to Amiga OS, but that results
still need to be re-integrated into the source.
Ports for Windows'95 and Windows NT exist, but are not yet
integrated with the main distribution.

\smallskip\noindent
Thanks go to \Hennecke, \AxelRamge, \Rouchal{} and \Wortmann{} for
fruitful and refreshening discussions about that \fn{Override.pm}
loading scheme (which finally made its way after enough chickens and
eggs chased one another to death \mbox{$\mathsmiley$}\,),
and to \Taupin{} for his successful efforts to get \latextohtml{}
running on DOS.

\smallskip\noindent
Thanks go also to \Popineau{} for his port to Windows NT
\footnote{\ctanURL{systems/win32/web2c/l2h-win32.tar.gz}},
and \NikosDrakos{} for a Windows 95 port based on \textsc{v96.1}h
\footnote{\url{ftp://ftp.mpn.com/pub/nikos/latex2html96.1-h-win32.tar.gz}}
(which is mentioned here at last, but not least).

\medskip\noindent
We want to take the opportunity to thank \Nelson{} and the people at
Lawrence Livermore National Laboratory who help to keep up the
\latextohtml{} main archive and the mailing list, and to \Bohnet{} at
the Max Planck Institut fuer extraterrestrische Physik, Garching for
maintaining the list's online archive.
Finally thanks and greetings to all people that contributed to this
release and have not been mentioned here...

\medskip\noindent
You all showed spirit and favour. Thank you for your efforts!

\medskip
\begin{flushright}
\dots\ and don't forget Jens and the LiPS team at Darmstadt!
\end{flushright}

\clearpage
\section*{1998 to 1999}\label{recent99}%
During this period large parts of \latextohtml{} have been overhauled and
compatibility with Perl 4 broken once and for all. The 99.2 release is the
first known to work out of the box on several UNIX systems as well as on 
Windows 95, 98, NT and OS/2. The number of supported \LaTeX\ packages is
bigger than ever.

\medskip\noindent
Thanks to \Perbandt{} for testing every second alpha/beta release of
99.2 on OS/2 and ensuring that things work ok there.


\clearpage
\subsection*{Proposals for Future Development:\label{future}}%



\subsection*{\latextohtmlNG}
Developed by \Hennecke\label{latex2htmlNG} this is a version of 
\latextohtml{} that addresses various issues,
not currently handled in the best way by version \textsc{v97.1}\,.
These include:
\begin{itemize}
\item validating the \texttt{HTML} output,
so that only correctly nested tags, and their contents,
can be produced by the translator;
%
\item more \TeX-like order of macro-expansion,
so that macros and their expansions will produce exactly
the results expected from the \TeX{} implementation of \LaTeX;
%
\item faster processing,
by streamlining some of the current \Perl{} code, and allowing
shorter strings to be handled at any given time;
%
\item customisation issues,
allowing easier portability to Unix-like environments on
other platforms.
%
\end{itemize}
Many of these features have been the inspiration for new code
written for \latextohtml~\textsc{v98.1}.

\medskip\noindent
The current version of \latextohtmlNG{} can be obtained from
the developer's
\hyperref[page]{repository}{repository, see page~}{}{cvsrepos},
in the directory
\url{http://saftsack.fs.uni-bayreuth.de/~latex2ht/ng-user}.\html{\\}
Beware that the files there are \emph{not} compatible with those of the
same name that come with the current version of \latextohtml.



\subsection*{Extended Capabilities in Web browsers}
The following areas are the subject of active development
within the Web community. 
Limited support is available within \latextohtml{} for some of these features,
using the \texttt{-html\_version 4.0} command-line switch.
\begin{description}
%
\index{style sheets!CSS}%
\index{style sheets!DSSSL}%
\item [style-sheets: ] \htmladdnormallink{proposals}%
{http://www.w3.org/pub/WWW/TR/WD-style-970324.html}
for a flexible mechanism to allow cascading (CSS) and DSSSL, 
within \htmladdnormallinkfoot{\HTMLiv}{http://www.w3.org/pub/Markup/}.
%
\index{extended markup!XML}%
\index{XML!extended markup}%
\item [XML: ] \htmladdnormallinkfoot{eXtensible Markup Language}%
{http://www.w3.org/pub/WWW/TR/WD-xml.html}.

\index{mathematics!markup, MathML}%
\index{MathML!mathematics markup}%
\item [MathML: ] \htmladdnormallinkfoot{Mathematical Markup Language}%
{http://www.w3.org/pub/WWW/TR/WD-math-970515}.

\index{chemical!markup, CML}%
\index{CML!chemical markup}%
\item [CML: ] \htmladdnormallinkfoot{Chemical Markup Language}%
{http://www.venus.co.uk/omf/cml}.

\index{fonts!non-standard encodings}%
\item [Fonts: ] further support for non-standard font encodings.

\index{icons}%
\item [Icons: ] Alternative sets of icons for navigation buttons 
and other purposes.
\end{description}
For some background on these technologies read
\Goossens' survey article ``Hyper-activity in the Web-world''
in \textsl{CERN Computer Newsletter} 
\htmladdnormallinkfoot{No. 227}{http://wwwinfo.cern.ch/cnls/227/art\_xml.html},
and browse \AxelRamge's 
\htmladdnormallinkfoot{site}{http://www.ramge.de/ax/latex2html/latex2html.html}
for ideas on how they could be used with \latextohtml.



\endinput















